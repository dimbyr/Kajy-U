\chapter{Structures de donn\'ees fondamentales}
\minitoc
%\section{La mémoire}


\section{Variables et langage de programmation}

Avant de passer à la définition des variables, mettons-les en relation avec d'anciennes équations mathématiques. Nous avons tous résolu de nombreuses équations mathématiques depuis notre enfance. À titre d'exemple, considérons l'équation ci-dessous :

\begin{equation*}
	3x + 4y = 5
\end{equation*}

Nous n’avons pas à nous soucier de l’utilisation de cette équation. La chose importante que nous devons comprendre est que l’équation a des noms ($x$ et $y$) qui contiennent des valeurs (données). Cela signifie que les noms ($x$ et $y$) sont des espaces réservés pour représenter les données. De même, en programmation informatique, nous avons besoin de quelque chose pour conserver les données, et les variables sont le moyen d'y parvenir.

\section{Pointeurs}
\section{Types de données}
\section{Tableaux}
\section{Chaînes de caractères}
\section{Struct}

\section{Exercices}
