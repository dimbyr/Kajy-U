\chapter*{Introduction}

\noindent ``En fait, je dirai que la différence entre un mauvais programmeur et un bon réside dans le fait qu'il considère son code ou ses structures de données comme plus importants. Les mauvais programmeurs se soucient du code. Les bons programmeurs se soucient des structures de données et de leurs relations.''

\hfill - Linus Torvalds

\begin{center}
	----------------o0o----------------
\end{center}

Bienvenue dans ce cours sur les structures de données et les algorithmes à Kajy University, Antananarivo, Madagascar. Ce cours est conçu pour les étudiants de première année qui souhaitent acquérir une solide compréhension des concepts fondamentaux en informatique. Les structures de données et les algorithmes constituent le fondement de la programmation et du développement logiciel, essentiels pour toute carrière en informatique ou en ingénierie logicielle.

L'objectif principal de ce cours est de vous familiariser avec les différentes structures de données, telles que les tableaux, les listes chaînées, les piles, les files, les arbres binaires, et les graphes. Vous apprendrez également les algorithmes associés à ces structures, tels que les algorithmes de tri, de recherche, et de parcours. Le cours abordera également la complexité algorithmique, un élément crucial pour évaluer l'efficacité des algorithmes.

Les structures de données et les algorithmes sont au cœur de nombreux aspects de l'informatique. Ils permettent de stocker, organiser, et manipuler des données de manière efficace. Comprendre ces concepts vous aidera à résoudre des problèmes complexes, à développer des programmes performants, et à améliorer vos compétences en programmation.

Voici ce que vous apprendrez au cours de ce semestre :

\begin{itemize}
	\item \textbf{Structures de données fondamentales} : Vous apprendrez les concepts de base comme les variables, les pointeurs, les types de données, les tableaux, et les chaînes de caractères. Cela servira de fondation pour comprendre les structures de données avancées et les algorithmes.
	
	\item \textbf{Algorithmes de recherche et de tri} : Cette partie couvre les algorithmes courants de tri et de recherche, y compris le tri à bulles, le tri par insertion, le tri rapide, la recherche linéaire, et la recherche binaire. Ces algorithmes sont essentiels pour manipuler et organiser des données.
	
	\item \textbf{Complexité algorithmique} : Vous découvrirez les concepts de complexité algorithmique, y compris la notation $O$, $\Omega$ et $\Theta$. Vous étudierez également la récursion et comprendrez comment elle affecte la complexité algorithmique.
	
	\item \textbf{Structures de données avancées} : Cette section traite des structures de données avancées comme les listes chaînées, les piles, les files, les arbres binaires, et les graphes. Vous apprendrez également des algorithmes de parcours et leurs applications.
	
	\item \textbf{Applications d'algorithmes en IA} : Cette dernière partie explore les applications pratiques des structures de données et des algorithmes dans le domaine de l'intelligence artificielle. Vous étudierez des algorithmes comme la régression linéaire, la régression logistique, et le clustering K-means. Ces algorithmes constituent une base fondamentale pour comprendre comment l'intelligence artificielle utilise les algorithmes pour résoudre des problèmes complexes.
\end{itemize}

Ce cours adopte une approche pratique. Vous aurez des sessions théoriques pour comprendre les concepts de base, suivies de travaux pratiques pour appliquer ces concepts dans des exercices et des projets. Vous travaillerez avec le langage C pour mettre en œuvre des structures de données et des algorithmes, et développer des compétences en programmation.

À la fin du cours, vous aurez une compréhension solide des structures de données et des algorithmes, et vous serez en mesure de les appliquer dans des contextes réels. Ces compétences vous seront utiles tout au long de vos études et de votre carrière professionnelle.

Nous sommes impatients de vous accompagner dans ce voyage d'apprentissage et de vous voir progresser dans le domaine passionnant des structures de données et des algorithmes. Bon cours !
