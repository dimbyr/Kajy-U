\chapter{Algorithmes de recherche et de tri}

\section{Qu’est-ce qu’un algorithme ?}

Un \emph{algorithme} est une série d'étapes visant à résoudre un problème ou à accomplir une tâche spécifique. En informatique, les algorithmes permettent de manipuler, trier, rechercher, et transformer des données. Pour qu'un algorithme soit efficace, il doit être:

\begin{enumerate}
	\item \textbf{Correct}: Donner le bon résultat pour tous les cas.
	\item \textbf{Efficace}: Rapide et utilisant des ressources raisonnables.
	\item \textbf{Simple}: Compréhensible et facile à mettre en œuvre.
	\item \textbf{Flexible}: Adaptable à différentes situations.
\end{enumerate}

Un exemple simple d'algorithme qui trouve le maximum de deux nombres:

\begin{algorithm}
	\caption{Trouver le maximum de deux nombres}
	\begin{algorithmic}
		\If {a > b}
		\State retour a
		\Else
		\State retour b
		\EndIf
	\end{algorithmic}
\end{algorithm}

\section{Récursion}

La récursion est un concept fondamental en informatique où une fonction s'appelle elle-même. C'est souvent utilisé pour résoudre des problèmes qui peuvent être décomposés en sous-problèmes similaires. Les fonctions récursives nécessitent une condition de terminaison pour éviter des appels infinis, ce qui pourrait entraîner des débordements de pile et des erreurs critiques.

Un exemple classique de récursion est le calcul de la factorielle d'un nombre. Voici un algorithme qui montre comment la récursion fonctionne pour la factorielle:

\begin{algorithm}
	\caption{Calcul de la factorielle d'un nombre}
	\begin{algorithmic}
		\Function{factorielle}{n}
		\If{n == 0}
		\State \Return 1
		\Else
		\State \Return n $\times$ \Call{factorielle}{n - 1}
	    \EndIf
		\EndFunction
	\end{algorithmic}
\end{algorithm}

Dans cet exemple, la fonction `factorielle` utilise la récursion pour multiplier un nombre par la factorielle du nombre précédent, jusqu'à ce qu'elle atteigne le cas de base (n == 0). La condition de terminaison empêche la récursion infinie et garantit que la fonction finit par retourner une valeur.

Les fonctions récursives peuvent entraîner des complexités spatiales élevées en raison de l'utilisation de la pile pour stocker les appels récursifs. Chaque appel de fonction récursive crée un nouveau contexte d'exécution dans la pile, ce qui peut augmenter l'utilisation de la mémoire. Pour éviter des débordements de pile, il est essentiel d'avoir des conditions de terminaison robustes et de gérer la profondeur de récursion.

\subsection*{Exemples d'autres problèmes résolus par la récursion}
\textit{Calculer la suite de Fibonacci}: La suite de Fibonacci peut être calculée récursivement en additionnant les deux termes précédents, avec des cas de base pour les premiers termes. Rapellons la d\'efinition de la suite de Fibonacci: 
\[ \begin{cases}
	U_0 = U_1 = 1 & \\
	U_{n+2} = U_{n+1} + U_{n} & \text{pour } n\ge 0.
\end{cases} \]


La récursion est un outil puissant pour résoudre des problèmes de manière élégante, mais elle doit être utilisée avec précaution pour éviter des complications liées à la mémoire et à la performance.

\section{Algorithmes de tri}

Le \emph{tri} consiste à organiser les données dans un ordre particulier. Les algorithmes de tri couramment utilisés comprennent:

\begin{enumerate}
		\item \textbf{Tri à bulles (Bubble Sort)} : Cet algorithme compare des éléments adjacents (côte à côte) et les échange s'ils ne sont pas dans le bon ordre. On répète le processus jusqu'à ce que toute la liste soit triée. C'est le plus simple des algorithmes de tri, mais il n'est pas très efficace pour les grandes listes, car il doit comparer de nombreux éléments.
	
%	Algorithm \ref{bulles} donne  un exemple de tri à bulles. 
	\item \textbf{Tri par insertion (Insertion Sort)} : Cet algorithme place chaque élément à la bonne position dans la liste, un par un. C'est efficace pour les petites listes ou les listes presque triées, car il insère chaque nouvel élément à sa position correcte sans avoir besoin de beaucoup de déplacements.

	
	\item \textbf{Tri rapide (QuickSort)} : Cet algorithme utilise la technique de division et conquête. Il choisit un pivot (un élément central), divise la liste en deux parties (éléments plus petits d'un côté, plus grands de l'autre), puis trie chaque partie séparément. Cela en fait un algorithme rapide et efficace pour les grandes listes.

	
	\item \textbf{Tri fusion (MergeSort)} : Cet algorithme divise la liste en deux moitiés, trie chaque moitié séparément, puis les fusionne pour créer une liste triée. C'est un tri stable, ce qui signifie qu'il maintient l'ordre relatif des éléments égaux. C'est souvent utilisé pour les grandes listes car il offre de bonnes performances.

	
%	\item \textbf{Tri par tas (HeapSort)}: Utilise une structure de tas (heap) pour trier les éléments. Il crée un tas binaire et extrait le plus grand élément pour le placer à la bonne position.
\end{enumerate}


\begin{algorithm}
	\caption{Tri à bulles}
	\label{bulles}
	\begin{algorithmic}[1]
		\While{non triée}
		\For{i de 0 à n - 2}
		\For{j de 0 à n - i - 2}
		\If{tableau[j] > tableau[j + 1]}
		\State échanger tableau[j] et tableau[j + 1]
		\EndIf
		\EndFor
		\EndFor
		\EndWhile
	\end{algorithmic}
\end{algorithm}

\begin{algorithm}
	\caption{Tri par insertion}
	\begin{algorithmic}[1]
		\For{i de 1 à n - 1}
		\State clé := tableau[i]
		\State j := i - 1
		\While{j >= 0 et tableau[j] > clé}
		\State tableau[j + 1] = tableau[j]
		\State j = j - 1
		\EndWhile
		\State tableau[j + 1] = clé
		\EndFor
	\end{algorithmic}
\end{algorithm}

\begin{algorithm}
	\caption{Tri rapide (QuickSort)}
	\begin{algorithmic}[1]
		\Function{QuickSort}{tableau, bas, haut}
		\If{bas < haut}
		\State pivot := \Call{Partition}{tableau, bas, haut}
		\State \Call{QuickSort}{tableau, bas, pivot - 1}
		\State \Call{QuickSort}{tableau, pivot + 1, haut}
		\EndIf
		\EndFunction
	\end{algorithmic}
\end{algorithm}

\begin{algorithm}
	\caption{Tri fusion (MergeSort)}
	\begin{algorithmic}[1]
		\Function{MergeSort}{tableau, gauche, droite}
		\If{gauche < droite}
		\State milieu := (gauche + droite) / 2
		\State \Call{MergeSort}{tableau, gauche, milieu}
		\State \Call{MergeSort}{tableau, milieu + 1, droite}
		\State \Call{Fusion}{tableau, gauche, milieu, droite}
		\EndIf
		\EndFunction
	\end{algorithmic}
\end{algorithm}

\begin{algorithm}
	\caption{Tri par tas (HeapSort)}
	\begin{algorithmic}[1]
		\Function{HeapSort}{tableau}
		\State Construire le tas
		\For{i de n - 1 à 1}
		\State échanger tableau[0] et tableau[i]
		\State Réorganiser le tas pour maintenir la propriété du tas
		\EndFor
		\EndFunction
	\end{algorithmic}
\end{algorithm}

		Les algorithmes de tri sont essentiels pour organiser les données efficacement. Chacun de ces algorithmes a ses avantages et inconvénients, avec des applications spécifiques en fonction des contraintes de temps et d'espace. Le choix de l'algorithme de tri dépendra de nombreux facteurs, y compris la taille de la liste, les caractéristiques des données, et les exigences de performance. 
		
\textit{T\^ache: Montrer que chacune des algorithmes de tri ci-dessus se termine.
}

\section{Algorithmes de recherche}

Les algorithmes de recherche permettent de trouver des éléments spécifiques dans un ensemble de données. Voici quelques-uns des types courants d'algorithmes de recherche, avec des explications détaillées et des exemples:

\begin{enumerate}
	\item \textbf{Recherche linéaire (Linear Search)} : Ce type de recherche parcourt une liste d'éléments un par un, de manière séquentielle, jusqu'à ce qu'il trouve l'élément recherché ou atteigne la fin de la liste. C'est le plus simple des algorithmes de recherche, mais il peut être lent pour les grandes listes.
	
%	\textbf{Exemple de recherche linéaire}

	\item \textbf{Recherche binaire (Binary Search)} : La recherche binaire utilise une technique de division pour trouver rapidement un élément dans une liste triée. Elle divise la liste en moitiés, recherche le milieu, puis réduit le domaine de recherche en fonction de la valeur recherchée. Cela la rend plus rapide que la recherche linéaire, mais elle nécessite que la liste soit triée.
	
%	\textbf{Exemple de recherche binaire}
	
	
	\item \textbf{Recherche par hachage (Hash Table Search)} : Ce type de recherche utilise une table de hachage pour trouver rapidement des éléments. Une fonction de hachage convertit chaque élément en une clé unique pour le stockage et la recherche. C'est très rapide pour les grandes listes, mais peut être complexe à mettre en place.
	
	\item \textbf{Recherche dans un arbre binaire (Binary Tree Search)} : Cette recherche utilise une structure appelée arbre binaire, où chaque nœud a jusqu'à deux enfants. La recherche se fait en parcourant les nœuds selon des règles spécifiques, ce qui peut être efficace pour certaines opérations de recherche.
	
	\item \textbf{Recherche interpolée (Interpolation Search)} : Cette recherche est une version améliorée de la recherche binaire pour les listes avec des valeurs bien réparties. Elle utilise une formule pour estimer la position de la valeur recherchée, ce qui la rend plus précise.
	
%	\textbf{Exemple de recherche interpolée}
	
	
	\item \textbf{Recherche par saut (Jump Search)} : C'est une recherche linéaire optimisée pour les grandes listes. Elle saute de blocs en blocs pour trouver la plage dans laquelle l'élément pourrait se trouver, puis fait une recherche linéaire à l'intérieur de cette plage.
	
%	\textbf{Exemple de recherche par saut}


\section{Exercices}

Pour pratiquer les algorithmes de recherche, voici quelques exercices qui vous aideront à comprendre et à appliquer ces concepts:

\begin{enumerate}
	\item \textbf{Recherche linéaire} : Implémentez une recherche linéaire pour trouver un élément spécifique dans un tableau. Quelles sont les limites de cette méthode ?
	\item \textbf{Recherche binaire} : Implémentez une recherche binaire sur un tableau trié. Que se passe-t-il si le tableau n'est pas trié ?
	\item \textbf{Recherche par hachage} : Expliquez comment une table de hachage peut accélérer la recherche. Quels sont les risques associés à cette méthode ?
	\item \textbf{Recherche interpolée} : Pourquoi la recherche interpolée est-elle plus efficace que la recherche binaire pour les listes avec des valeurs bien réparties ? Implémentez cet algorithme et montrez ses avantages.
	\item \textbf{Recherche par saut} : Implémentez une recherche par saut sur un tableau de grande taille. Pourquoi cette méthode peut-elle être plus rapide que la recherche linéaire pour des tableaux de grande taille ?
\end{enumerate}


	\begin{algorithm}
	\caption{Recherche linéaire}
	\begin{algorithmic}[1]
		\For{i de 0 à n - 1}
		\If{tableau[i] == cible}
		\State \textbf{Retourner} i
		\EndIf
		\EndFor
		\State \textbf{Retourner} -1
	\end{algorithmic}
\end{algorithm}

\begin{algorithm}
	\caption{Recherche binaire}
	\begin{algorithmic}
		\State gauche := 0
		\State droite := n - 1
		\While{gauche <= droite}
		\State milieu := gauche + (droite - gauche) / 2
		\If{tableau[milieu] == cible}
		\State \textbf{Retourner} milieu
		\ElsIf{tableau[milieu] < cible}
		\State gauche := milieu + 1
		\Else
		\State droite := milieu - 1
		\EndIf
		\EndWhile
		\State \textbf{Retourner} -1
	\end{algorithmic}
\end{algorithm}

\begin{algorithm}
	\caption{Recherche interpolée}
	\begin{algorithmic}[1]
		\State bas := 0
		\State haut := n - 1
		\While{bas <= haut et cible >= tableau[bas] et cible <= tableau[haut]}
		\State pos := bas + ((cible - tableau[bas]) $\div$ (tableau[haut] - tableau[bas])) $\times$ (haut - bas)
		\If{tableau[pos] == cible}
		\State \textbf{Retourner} pos
		\ElsIf{tableau[pos] < cible}
		\State bas := pos + 1
		\Else
		\State haut := pos - 1
		\EndIf
		\EndWhile
		\State \textbf{Retourner} -1
	\end{algorithmic}
\end{algorithm}


\begin{algorithm}
	\caption{Recherche par saut}
	\begin{algorithmic}[1]
		\State tailleBloc := \textbf{Racine} n
		\State i := 0
		\While{i < n et tableau[i] < cible}
		\State i := i + tailleBloc
		\EndWhile
		\For{j de i - tailleBloc à \textbf{Min}(i, n)}
		\If{tableau[j] == cible}
		\State \textbf{Retourner} j
		\EndIf
		\EndFor
		\State \textbf{Retourner} -1
	\end{algorithmic}
\end{algorithm}
\end{enumerate}


%\textit{\textbf{Objectifs des exercices
%}}
%Ces exercices vous aideront à pratiquer et à comprendre différents types d'algorithmes de recherche. Vous allez travailler avec des tableaux, des fonctions de recherche, et apprendre à évaluer les avantages et inconvénients de chaque algorithme. Le but est de vous donner une base solide pour comprendre comment ces algorithmes sont utilisés dans des applications réelles et pourquoi ils sont importants en informatique.

