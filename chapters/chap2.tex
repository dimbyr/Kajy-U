\chapter{Algorithmes de recherche et de tri}
\section{Qu’est-ce qu’un algorithme?}
\begin{definition}{Algorithme} a
	Un \emph{algorithme} est constitué d’instructions étape par étape sans ambiguïté pour résoudre un problème donné.
\end{definition}
\begin{example}
	Considérons le problème de la préparation d'une omelette. Pour préparer une omelette, on suit la
	étapes indiquées ci-dessous :
	\begin{enumerate}
		\item Récupérez la poêle.
		\item  Récupérez l'huile.
		
		\begin{enumerate}
			\item Avons-nous de l'huile?
	 \begin{enumerate}
	 	\item Si oui, mettez-le dans la poêle.
	 \item  Si non, voulons-nous acheter de l'huile ?
	 \begin{enumerate}
	 	\item Si oui, sortez et achetez.
	 	\item Si non, on termine.
	 
	 \end{enumerate}
		\end{enumerate}
	 \end{enumerate}
		\item Allumez la cuisinière, etc...
	\end{enumerate}
	Ce que nous faisons, c'est que, pour un problème donné (préparer une omelette), nous proposons une procédure étape par étape pour le résoudre.
\end{example}
\section{Tri}
\subsection{Tri \`a bulle}
\subsection{Tri par tas}

\section{Algorithmes de recherche}
\subsection{Recherche syst\'ematique}
\subsection{Recherche par }

\section{Exercices}
