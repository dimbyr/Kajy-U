% !TeX spellcheck = en_US
\documentclass[11pt,fancy,authorye]{elegantbook}

% Force chapter numbering
\renewcommand{\chaptermark}[1]{%
	\markboth{\thechapter.\ #1}{}}
\titleformat{\chapter}[display]
{\normalfont\huge\bfseries}{\chaptername\ \thechapter}{20pt}{\Huge}

% Ensure numbering depth
\setcounter{secnumdepth}{1} % Number chapters and sections
\setcounter{tocdepth}{2}    % Ensure depth in TOC

\usepackage{xcolor}
\usepackage[utf8]{inputenc}
\usepackage[french]{babel}
\usepackage{algorithm}
\usepackage{algpseudocode}

% Custom definitions for French keywords in algorithmicx
\algrenewcommand\algorithmicif{\textbf{Si}}
\algrenewcommand\algorithmicelse{\textbf{Sinon}}
\algrenewcommand\algorithmicfor{\textbf{Pour}}
\algrenewcommand\algorithmicdo{\textbf{Faire}}
\algrenewcommand\algorithmicwhile{\textbf{Tant~que}}
\algrenewcommand\algorithmicthen{\textbf{Alors}}
\algrenewcommand\algorithmicend{\textbf{Fin}}
\algrenewcommand\algorithmicreturn{\textbf{Retour}}
\usepackage[utf8]{inputenc} % For UTF-8 encoding
\usepackage[T1]{fontenc}
\usepackage{minitoc}
\usepackage{mdframed}
\usepackage{verbatim}
\usepackage{listings}
\usepackage{xcolor}  % To use colors

\lstset{% Formatting C codes
	language=C,
	basicstyle=\ttfamily,
	inputencoding=utf8,      % Ensure UTF-8 encoding for code
	extendedchars=true,      % Enable extended characters
	literate={é}{{\'e}}1,    % Map 'é' to its LaTeX representation
	numbers=left,
	xleftmargin=1.5cm,
	xrightmargin=.5cm,
	showstringspaces=false,
	backgroundcolor=\color{lightgray}  % Set background color to light gray
}

\definecolor{remarkcolor}{RGB}{255,128,0} % Define the color for "Remark"

\newtheorem{remarque}{Remarque}

\title{Kajy University: Informatique}
\subtitle{Algorithmes et structures de données}

\author{Dimby Rabearivony}
\date{\today}
\version{1.1}

\extrainfo{"Ny fahalalana no valamparihiko."}

\logo{Kajy U.png}
\cover{hay.jpg}

\begin{document}
	
	\maketitle
	
	\frontmatter
	\tableofcontents
	
	\mainmatter
	\chapter*{Introduction}

\noindent ``En fait, je dirai que la différence entre un mauvais programmeur et un bon réside dans le fait qu'il considère son code ou ses structures de données comme plus importants. Les mauvais programmeurs se soucient du code. Les bons programmeurs se soucient des structures de données et de leurs relations.''

\hfill - Linus Torvalds % Chapitre 0
	\chapter{Structures de donn\'ees fondamentales}
\chaptertoc
\section{The memory}
\section{La mémoire}
\section{Variables et langage de programmation}
\section{Types de données}
\section{Tableaux}
\section{Chaînes de caractères}
\section{Struct} % Chapitre 1
	\chapter{Algorithmes de recherche et de tri}

\section{Qu’est-ce qu’un algorithme ?}

Un \emph{algorithme} est une série d'étapes visant à résoudre un problème ou à accomplir une tâche spécifique. En informatique, les algorithmes permettent de manipuler, trier, rechercher, et transformer des données. Pour qu'un algorithme soit efficace, il doit être:

\begin{enumerate}
	\item \textbf{Correct}: Donner le bon résultat pour tous les cas.
	\item \textbf{Efficace}: Rapide et utilisant des ressources raisonnables.
	\item \textbf{Simple}: Compréhensible et facile à mettre en œuvre.
	\item \textbf{Flexible}: Adaptable à différentes situations.
\end{enumerate}

Un exemple simple d'algorithme qui trouve le maximum de deux nombres:

\begin{algorithm}
	\caption{Trouver le maximum de deux nombres}
	\begin{algorithmic}
		\If {a > b}
		\State retour a
		\Else
		\State retour b
		\EndIf
	\end{algorithmic}
\end{algorithm}

\section{Récursion}

La récursion est un concept fondamental en informatique où une fonction s'appelle elle-même. C'est souvent utilisé pour résoudre des problèmes qui peuvent être décomposés en sous-problèmes similaires. Les fonctions récursives nécessitent une condition de terminaison pour éviter des appels infinis, ce qui pourrait entraîner des débordements de pile et des erreurs critiques.

Un exemple classique de récursion est le calcul de la factorielle d'un nombre. Voici un algorithme qui montre comment la récursion fonctionne pour la factorielle:

\begin{algorithm}
	\caption{Calcul de la factorielle d'un nombre}
	\begin{algorithmic}
		\Function{factorielle}{n}
		\If{n == 0}
		\State \Return 1
		\Else
		\State \Return n $\times$ \Call{factorielle}{n - 1}
	    \EndIf
		\EndFunction
	\end{algorithmic}
\end{algorithm}

Dans cet exemple, la fonction `factorielle` utilise la récursion pour multiplier un nombre par la factorielle du nombre précédent, jusqu'à ce qu'elle atteigne le cas de base (n == 0). La condition de terminaison empêche la récursion infinie et garantit que la fonction finit par retourner une valeur.

Les fonctions récursives peuvent entraîner des complexités spatiales élevées en raison de l'utilisation de la pile pour stocker les appels récursifs. Chaque appel de fonction récursive crée un nouveau contexte d'exécution dans la pile, ce qui peut augmenter l'utilisation de la mémoire. Pour éviter des débordements de pile, il est essentiel d'avoir des conditions de terminaison robustes et de gérer la profondeur de récursion.

\subsection*{Exemples d'autres problèmes résolus par la récursion}
\textit{Calculer la suite de Fibonacci}: La suite de Fibonacci peut être calculée récursivement en additionnant les deux termes précédents, avec des cas de base pour les premiers termes. Rapellons la d\'efinition de la suite de Fibonacci: 
\[ \begin{cases}
	U_0 = U_1 = 1 & \\
	U_{n+2} = U_{n+1} + U_{n} & \text{pour } n\ge 0.
\end{cases} \]


La récursion est un outil puissant pour résoudre des problèmes de manière élégante, mais elle doit être utilisée avec précaution pour éviter des complications liées à la mémoire et à la performance.

\section{Algorithmes de tri}

Le \emph{tri} consiste à organiser les données dans un ordre particulier. Les algorithmes de tri couramment utilisés comprennent:

\begin{enumerate}
		\item \textbf{Tri à bulles (Bubble Sort)} : Cet algorithme compare des éléments adjacents (côte à côte) et les échange s'ils ne sont pas dans le bon ordre. On répète le processus jusqu'à ce que toute la liste soit triée. C'est le plus simple des algorithmes de tri, mais il n'est pas très efficace pour les grandes listes, car il doit comparer de nombreux éléments.
	
%	Algorithm \ref{bulles} donne  un exemple de tri à bulles. 
	\item \textbf{Tri par insertion (Insertion Sort)} : Cet algorithme place chaque élément à la bonne position dans la liste, un par un. C'est efficace pour les petites listes ou les listes presque triées, car il insère chaque nouvel élément à sa position correcte sans avoir besoin de beaucoup de déplacements.

	
	\item \textbf{Tri rapide (QuickSort)} : Cet algorithme utilise la technique de division et conquête. Il choisit un pivot (un élément central), divise la liste en deux parties (éléments plus petits d'un côté, plus grands de l'autre), puis trie chaque partie séparément. Cela en fait un algorithme rapide et efficace pour les grandes listes.

	
	\item \textbf{Tri fusion (MergeSort)} : Cet algorithme divise la liste en deux moitiés, trie chaque moitié séparément, puis les fusionne pour créer une liste triée. C'est un tri stable, ce qui signifie qu'il maintient l'ordre relatif des éléments égaux. C'est souvent utilisé pour les grandes listes car il offre de bonnes performances.

	
%	\item \textbf{Tri par tas (HeapSort)}: Utilise une structure de tas (heap) pour trier les éléments. Il crée un tas binaire et extrait le plus grand élément pour le placer à la bonne position.
\end{enumerate}


\begin{algorithm}
	\caption{Tri à bulles}
	\label{bulles}
	\begin{algorithmic}[1]
		\While{non triée}
		\For{i de 0 à n - 2}
		\For{j de 0 à n - i - 2}
		\If{tableau[j] > tableau[j + 1]}
		\State échanger tableau[j] et tableau[j + 1]
		\EndIf
		\EndFor
		\EndFor
		\EndWhile
	\end{algorithmic}
\end{algorithm}

\begin{algorithm}
	\caption{Tri par insertion}
	\begin{algorithmic}[1]
		\For{i de 1 à n - 1}
		\State clé := tableau[i]
		\State j := i - 1
		\While{j >= 0 et tableau[j] > clé}
		\State tableau[j + 1] = tableau[j]
		\State j = j - 1
		\EndWhile
		\State tableau[j + 1] = clé
		\EndFor
	\end{algorithmic}
\end{algorithm}

\begin{algorithm}
	\caption{Tri rapide (QuickSort)}
	\begin{algorithmic}[1]
		\Function{QuickSort}{tableau, bas, haut}
		\If{bas < haut}
		\State pivot := \Call{Partition}{tableau, bas, haut}
		\State \Call{QuickSort}{tableau, bas, pivot - 1}
		\State \Call{QuickSort}{tableau, pivot + 1, haut}
		\EndIf
		\EndFunction
	\end{algorithmic}
\end{algorithm}

\begin{algorithm}
	\caption{Tri fusion (MergeSort)}
	\begin{algorithmic}[1]
		\Function{MergeSort}{tableau, gauche, droite}
		\If{gauche < droite}
		\State milieu := (gauche + droite) / 2
		\State \Call{MergeSort}{tableau, gauche, milieu}
		\State \Call{MergeSort}{tableau, milieu + 1, droite}
		\State \Call{Fusion}{tableau, gauche, milieu, droite}
		\EndIf
		\EndFunction
	\end{algorithmic}
\end{algorithm}

\begin{algorithm}
	\caption{Tri par tas (HeapSort)}
	\begin{algorithmic}[1]
		\Function{HeapSort}{tableau}
		\State Construire le tas
		\For{i de n - 1 à 1}
		\State échanger tableau[0] et tableau[i]
		\State Réorganiser le tas pour maintenir la propriété du tas
		\EndFor
		\EndFunction
	\end{algorithmic}
\end{algorithm}

		Les algorithmes de tri sont essentiels pour organiser les données efficacement. Chacun de ces algorithmes a ses avantages et inconvénients, avec des applications spécifiques en fonction des contraintes de temps et d'espace. Le choix de l'algorithme de tri dépendra de nombreux facteurs, y compris la taille de la liste, les caractéristiques des données, et les exigences de performance. 
		
\textit{T\^ache: Montrer que chacune des algorithmes de tri ci-dessus se termine.
}

\section{Algorithmes de recherche}

Les algorithmes de recherche permettent de trouver des éléments spécifiques dans un ensemble de données. Voici quelques-uns des types courants d'algorithmes de recherche, avec des explications détaillées et des exemples:

\begin{enumerate}
	\item \textbf{Recherche linéaire (Linear Search)} : Ce type de recherche parcourt une liste d'éléments un par un, de manière séquentielle, jusqu'à ce qu'il trouve l'élément recherché ou atteigne la fin de la liste. C'est le plus simple des algorithmes de recherche, mais il peut être lent pour les grandes listes.
	
%	\textbf{Exemple de recherche linéaire}

	\item \textbf{Recherche binaire (Binary Search)} : La recherche binaire utilise une technique de division pour trouver rapidement un élément dans une liste triée. Elle divise la liste en moitiés, recherche le milieu, puis réduit le domaine de recherche en fonction de la valeur recherchée. Cela la rend plus rapide que la recherche linéaire, mais elle nécessite que la liste soit triée.
	
%	\textbf{Exemple de recherche binaire}
	
	
	\item \textbf{Recherche par hachage (Hash Table Search)} : Ce type de recherche utilise une table de hachage pour trouver rapidement des éléments. Une fonction de hachage convertit chaque élément en une clé unique pour le stockage et la recherche. C'est très rapide pour les grandes listes, mais peut être complexe à mettre en place.
	
	\item \textbf{Recherche dans un arbre binaire (Binary Tree Search)} : Cette recherche utilise une structure appelée arbre binaire, où chaque nœud a jusqu'à deux enfants. La recherche se fait en parcourant les nœuds selon des règles spécifiques, ce qui peut être efficace pour certaines opérations de recherche.
	
	\item \textbf{Recherche interpolée (Interpolation Search)} : Cette recherche est une version améliorée de la recherche binaire pour les listes avec des valeurs bien réparties. Elle utilise une formule pour estimer la position de la valeur recherchée, ce qui la rend plus précise.
	
%	\textbf{Exemple de recherche interpolée}
	
	
	\item \textbf{Recherche par saut (Jump Search)} : C'est une recherche linéaire optimisée pour les grandes listes. Elle saute de blocs en blocs pour trouver la plage dans laquelle l'élément pourrait se trouver, puis fait une recherche linéaire à l'intérieur de cette plage.
	
%	\textbf{Exemple de recherche par saut}


\section{Exercices}

Pour pratiquer les algorithmes de recherche, voici quelques exercices qui vous aideront à comprendre et à appliquer ces concepts:

\begin{enumerate}
	\item \textbf{Recherche linéaire} : Implémentez une recherche linéaire pour trouver un élément spécifique dans un tableau. Quelles sont les limites de cette méthode ?
	\item \textbf{Recherche binaire} : Implémentez une recherche binaire sur un tableau trié. Que se passe-t-il si le tableau n'est pas trié ?
	\item \textbf{Recherche par hachage} : Expliquez comment une table de hachage peut accélérer la recherche. Quels sont les risques associés à cette méthode ?
	\item \textbf{Recherche interpolée} : Pourquoi la recherche interpolée est-elle plus efficace que la recherche binaire pour les listes avec des valeurs bien réparties ? Implémentez cet algorithme et montrez ses avantages.
	\item \textbf{Recherche par saut} : Implémentez une recherche par saut sur un tableau de grande taille. Pourquoi cette méthode peut-elle être plus rapide que la recherche linéaire pour des tableaux de grande taille ?
\end{enumerate}


	\begin{algorithm}
	\caption{Recherche linéaire}
	\begin{algorithmic}[1]
		\For{i de 0 à n - 1}
		\If{tableau[i] == cible}
		\State \textbf{Retourner} i
		\EndIf
		\EndFor
		\State \textbf{Retourner} -1
	\end{algorithmic}
\end{algorithm}

\begin{algorithm}
	\caption{Recherche binaire}
	\begin{algorithmic}
		\State gauche := 0
		\State droite := n - 1
		\While{gauche <= droite}
		\State milieu := gauche + (droite - gauche) / 2
		\If{tableau[milieu] == cible}
		\State \textbf{Retourner} milieu
		\ElsIf{tableau[milieu] < cible}
		\State gauche := milieu + 1
		\Else
		\State droite := milieu - 1
		\EndIf
		\EndWhile
		\State \textbf{Retourner} -1
	\end{algorithmic}
\end{algorithm}

\begin{algorithm}
	\caption{Recherche interpolée}
	\begin{algorithmic}[1]
		\State bas := 0
		\State haut := n - 1
		\While{bas <= haut et cible >= tableau[bas] et cible <= tableau[haut]}
		\State pos := bas + ((cible - tableau[bas]) $\div$ (tableau[haut] - tableau[bas])) $\times$ (haut - bas)
		\If{tableau[pos] == cible}
		\State \textbf{Retourner} pos
		\ElsIf{tableau[pos] < cible}
		\State bas := pos + 1
		\Else
		\State haut := pos - 1
		\EndIf
		\EndWhile
		\State \textbf{Retourner} -1
	\end{algorithmic}
\end{algorithm}


\begin{algorithm}
	\caption{Recherche par saut}
	\begin{algorithmic}[1]
		\State tailleBloc := \textbf{Racine} n
		\State i := 0
		\While{i < n et tableau[i] < cible}
		\State i := i + tailleBloc
		\EndWhile
		\For{j de i - tailleBloc à \textbf{Min}(i, n)}
		\If{tableau[j] == cible}
		\State \textbf{Retourner} j
		\EndIf
		\EndFor
		\State \textbf{Retourner} -1
	\end{algorithmic}
\end{algorithm}
\end{enumerate}


%\textit{\textbf{Objectifs des exercices
%}}
%Ces exercices vous aideront à pratiquer et à comprendre différents types d'algorithmes de recherche. Vous allez travailler avec des tableaux, des fonctions de recherche, et apprendre à évaluer les avantages et inconvénients de chaque algorithme. Le but est de vous donner une base solide pour comprendre comment ces algorithmes sont utilisés dans des applications réelles et pourquoi ils sont importants en informatique.

 % Chapitre 2
	\chapter{Complexité algorithmique}

La complexité algorithmique permet d'évaluer l'efficacité des algorithmes en termes de temps d'exécution et de consommation de ressources. Cela aide à comprendre combien de temps un algorithme prendra pour des entrées de tailles différentes. Les notations couramment utilisées pour décrire la complexité algorithmique sont $O$, $\Omega$, et $\theta$.

\section{Les bornes asymptotiques}

Les bornes asymptotiques, ou "asymptotic bounds," fournissent une approximation de la performance d'un algorithme pour des tailles d'entrée importantes. Elles aident à évaluer le comportement des algorithmes sous différents scénarios : meilleur cas, pire cas, et cas moyen.

\begin{itemize}
	\item \textbf{Borne supérieure (Big O)} : Représentée par la notation \(O(f(n))\), cette borne donne une estimation de la complexité maximale d'un algorithme, indiquant qu'il n'aura jamais une complexité supérieure à \(O(f(n))\). Mathématiquement, cela signifie qu'il existe des constantes \(c\) et \(n_0\) telles que :
	
	\[ T(n) \leq c \cdot f(n), \quad \text{pour tout } n \geq n_0. \]
	
	\item \textbf{Borne inférieure (Big Omega)} : Représentée par la notation \(\Omega(f(n))\), cette borne donne une estimation de la complexité minimale, indiquant que l'algorithme ne peut pas être plus rapide que cette borne. Cela signifie qu'il existe des constantes \(c\) et \(n_0\) telles que :
	
	\[ T(n) \geq c \cdot f(n), \quad \text{pour tout } n \geq n_0. \]
	
	\item \textbf{Borne moyenne (Big Theta)} : Représentée par la notation \(\theta(f(n))\), cette borne indique une complexité moyenne ou attendue. Cela signifie que l'algorithme a une borne supérieure et inférieure qui convergent vers la même fonction. Mathématiquement, cela signifie qu'il existe des constantes \(c_1\), \(c_2\), et \(n_0\) telles que :
	
	\[ c_1 \cdot f(n) \leq T(n) \leq c_2 \cdot f(n), \quad \text{pour tout } n \geq n_0. \]
\end{itemize}

\section{La notation O}

La notation "O" (grand O) indique la complexité maximale d'un algorithme en termes de temps ou d'espace. Elle fournit une estimation du nombre maximum d'opérations nécessaires en fonction de la taille de l'entrée. Par exemple, la recherche binaire a une complexité de `O(log n)`, car chaque étape réduit le problème de moitié.

\begin{algorithm}
	\caption{Recherche binaire}
	\begin{algorithmic}[1]
		\Function{recherche\_binaire}{tableau, valeur}
		\State gauche := 0
		\State droite := taille(tableau) - 1
		\While{gauche <= droite}
		\State milieu := (gauche + droite) / 2
		\If{tableau[milieu] == valeur}
		\State \textbf{Retourner} milieu
		\ElsIf{tableau[milieu] < valeur}
		\State gauche := milieu + 1
		\Else
		\State droite := milieu - 1
		\EndIf
		\EndWhile
		\State \textbf{Retourner} -1
		\EndFunction
	\end{algorithmic}
\end{algorithm}

\section{Les notations $\Omega$ et $\theta$}

Ces notations complètent la notation "O" pour donner une vision complète de la complexité algorithmique.

\begin{itemize}
	\item \textbf{Borne inférieure (\(\Omega\))} : Représente la complexité minimale. Par exemple, la complexité de l'algorithme de somme de tableau est $\Omega(n)$, car chaque élément doit être parcouru.
	
	\item \textbf{Borne moyenne (\(\theta\))} : Indique la complexité moyenne ou attendue, définissant à la fois une borne inférieure et une borne supérieure qui convergent. Pour le même algorithme de somme de tableau, la complexité est $\theta(n)$. 
\end{itemize}

Voici un exemple de la somme des éléments d'un tableau :

\begin{algorithm}
	\caption{Somme de tableau}
	\begin{algorithmic}[1]
		\Function{somme}{tableau}
		\State somme := 0
		\For{i de 0 à taille(tableau) - 1}
		\State somme := somme + tableau[i]
		\EndFor
		\State \textbf{Retourner} somme
		\EndFunction
	\end{algorithmic}
\end{algorithm}

\section{Complexité de certains algorithmes}
Voici quelques exemples d'algorithmes qui illustrent des complexités courantes, allant du temps constant à l'exponentiel.

\subsection{Temps constant (O(1))}

Les algorithmes à temps constant ont une complexité qui ne dépend pas de la taille de l'entrée. Par exemple, accéder à un élément spécifique d'un tableau par son index a une complexité `O(1)`. Cela signifie qu'il y a toujours une seule opération, quelle que soit la taille du tableau.

\begin{algorithm}
	\caption{Accès direct à un élément du tableau}
	\begin{algorithmic}[1]
		\Function{accès\_direct}{tableau, index}
		\State \textbf{Retourner} tableau[index]
		\EndFunction
	\end{algorithmic}
\end{algorithm}

\subsection{Temps linéaire (O(n))}

Les algorithmes à temps linéaire ont une complexité qui augmente proportionnellement avec la taille de l'entrée. Par exemple, le calcul de la somme des éléments d'un tableau a une complexité `O(n)`, car il faut parcourir tous les éléments.

\begin{algorithm}
	\caption{Calcul de la somme des éléments d'un tableau}
	\begin{algorithmic}[1]
		\Function{somme}{tableau}
		\State somme := 0
		\For{i de 0 à taille(tableau) - 1}
		\State somme := somme + tableau[i]
		\EndFor
		\State \textbf{Retourner} somme
		\EndFunction
	\end{algorithmic}
\end{algorithm}

\subsection{Temps quadratique (O($n^2$))}

Les algorithmes à temps quadratique ont une complexité qui augmente avec le carré de la taille de l'entrée. Par exemple, le tri par insertion a une complexité `O($n^2$)`, car chaque élément doit être comparé avec tous les éléments précédents.

\begin{algorithm}
	\caption{Tri par insertion}
	\begin{algorithmic}[1]
		\For{i de 1 à n - 1}
		\State clé := tableau[i]
		\State j := i - 1
		\While{j >= 0 et tableau[j] > clé}
		\State tableau[j + 1] = tableau[j]
		\State j = j - 1
		\EndWhile
		\State tableau[j + 1] = clé
		\EndFor
	\end{algorithmic}
\end{algorithm}

\subsection{Temps exponentiel (O($2^n$))}

Les algorithmes à temps exponentiel ont une complexité qui augmente de manière exponentielle avec la taille de l'entrée. Par exemple, un algorithme récursif pour résoudre le problème de la tour de Hanoï a une complexité `O($2^n$)`, car chaque mouvement peut générer de multiples autres mouvements.

\begin{algorithm}
	\caption{Tour de Hanoï}
	\begin{algorithmic}[1]
		\Function{hanoi}{n, source, cible, intermédiaire}
		\If{n == 1}
		\State \textbf{Déplacer} un disque de la source vers la cible
		\Else
		\State \Call{hanoi}{n - 1, source, intermédiaire, cible}
		\State \textbf{Déplacer} un disque de la source vers la cible
		\State \Call{hanoi}{n - 1, intermédiaire, cible, source}
		\EndIf
		\EndFunction
	\end{algorithmic}
\end{algorithm}


%La complexité des algorithmes varie selon le nombre d'opérations effectuées, le type d'opérations, et la structure de l'algorithme. Examinons pourquoi certains algorithmes ont des complexités spécifiques et comment ces complexités sont dérivées.
%
%\begin{itemize}
%	\item \textbf{Tri par insertion (Insertion Sort)} : La complexité de cet algorithme est généralement `O($n^2$)`. Cela provient du fait que, pour chaque élément du tableau, l'algorithme doit le comparer avec tous les éléments précédents pour l'insérer au bon endroit. Ainsi, pour un tableau de taille \(n\), il y a jusqu'à \(n - 1\) comparaisons pour chaque élément, menant à une complexité quadratique.
%	
%	\item \textbf{Tri rapide (QuickSort)} : La complexité de cet algorithme est `O(n log n)`, ce qui le rend plus efficace que le tri par insertion pour des tableaux de grande taille. Cela provient de sa stratégie de division et conquête, où le tableau est divisé en sous-tableaux autour d'un pivot, puis chaque sous-tableau est trié séparément. À chaque division, le problème est réduit de moitié, ce qui génère une complexité logarithmique. Comme chaque élément doit être comparé au pivot, cela crée la partie linéaire de la complexité.
%	
%	\item \textbf{Recherche linéaire (Linear Search)} : La complexité de cet algorithme est `O(n)`. Cela signifie que pour trouver un élément, l'algorithme doit potentiellement parcourir tous les éléments du tableau. Si l'élément recherché est le dernier, tous les éléments précédents doivent être examinés, d'où la complexité linéaire.
%	
%	\item \textbf{Recherche binaire (Binary Search)} : La complexité de cet algorithme est `O(log n)`, ce qui le rend plus rapide que la recherche linéaire pour des tableaux triés. Cette complexité provient du fait que l'algorithme divise le tableau en deux à chaque itération. Ainsi, avec chaque division, le problème est réduit de moitié, ce qui conduit à une complexité logarithmique.
%	
%	\item \textbf{Tri par sélection (Selection Sort)} : Cet algorithme a également une complexité de `O($n^2$)`. Dans le tri par sélection, l'algorithme trouve le plus petit élément du tableau et le place au début, puis répète ce processus pour les éléments suivants. Le nombre de comparaisons nécessaires pour chaque élément est lié à la taille du tableau, ce qui explique la complexité quadratique.
%\end{itemize}
 % Chapitre 3
	%\chapter{Structures de données avancées}
%%\minitoc
%\section{Listes chaînées}
%\section{ Piles}
%\section{Files}
%\section{Arbres binaires}
%\section{Arbres}
%\section{ Graphes}
%
%\section{Exercices}


\chapter{Structures de données avancées}

\section{Listes chaînées}
Les listes chaînées sont des structures de données composées de nœuds, où chaque nœud contient des données et un pointeur vers le nœud suivant. Elles sont utiles pour des opérations d'insertion et de suppression rapides sans nécessiter de réallocation de mémoire.

Exemple d'implémentation d'une liste chaînée en C :

\begin{lstlisting}
	typedef struct Node {
		int data;
		struct Node* next;
	} Node;
	
	void insert(Node** head, int new_data) {
		Node* new_node = (Node*)malloc(sizeof(Node));
		new_node->data = new_data;
		new_node->next = *head;
		*head = new_node;
	}
	
	void delete(Node** head, int key) {
		Node* temp = *head, *prev;
		if (temp != NULL && temp->data == key) {
			*head = temp->next;
			free(temp);
			return;
		}
		while (temp != NULL && temp->data != key) {
			prev = temp;
			temp = temp->next;
		}
		if (temp == NULL) return;
		prev->next = temp->next;
		free(temp);
	}
\end{lstlisting}

\section{Piles}
Les piles (ou LIFO, Last In First Out) permettent des opérations de push et pop. Elles suivent le principe du dernier entré, premier sorti. Les piles sont couramment utilisées pour les opérations de retour en arrière, comme dans les navigateurs Web ou les algorithmes de récursion.

Voici un exemple simple d'utilisation de pile pour vérifier les parenthèses équilibrées :

\begin{lstlisting}
	int areParenthesesBalanced(char expr[]) {
		Stack* stack = createStack(strlen(expr));
		for (int i = 0; i < strlen(expr); i++) {
			if (expr[i] == '(') {
				push(stack, expr[i]);
			} else if (expr[i] == ')') {
				if (isEmpty(stack)) {
					return 0;
				}
				pop(stack);
			}
		}
		return isEmpty(stack);
	}
\end{lstlisting}

\section{Files}
Les files (ou FIFO, First In First Out) permettent des opérations d'enfilage et de défilage. Elles sont utiles pour la gestion des tâches, les files d'attente, et les algorithmes de parcours.

Exemple d'une file utilisant une structure de données basée sur des listes chaînées :

\begin{lstlisting}
	typedef struct QueueNode {
		int data;
		struct QueueNode* next;
	} QueueNode;
	
	typedef struct Queue {
		QueueNode* front;
		QueueNode* rear;
	} Queue;
	
	Queue* createQueue() {
		Queue* queue = (Queue*)malloc(sizeof(Queue));
		queue->front = queue->rear = NULL;
		return queue;
	}
	
	void enqueue(Queue* queue, int data) {
		QueueNode* new_node = (QueueNode*)malloc(sizeof(QueueNode));
		new_node->data = data;
		new_node->next = NULL;
		if (queue->rear == NULL) {
			queue->rear = queue->front = new_node;
		} else {
			queue->rear->next = new_node;
			queue->rear = new_node;
		}
	}
	
	int dequeue(Queue* queue) {
		if (queue->front == NULL) {
			return -1;
		}
		QueueNode* temp = queue->front;
		queue->front = queue->front->next;
		if (queue->front == NULL) {
			queue->rear = NULL;
		}
		int data = temp->data;
		free(temp);
		return data;
	}
\end{lstlisting}

\section{Arbres binaires}
Les arbres binaires sont des structures de données où chaque nœud a au maximum deux enfants, généralement appelés gauche et droite. Les arbres binaires sont utilisés pour des opérations de recherche et des algorithmes de parcours comme pré-ordre, en-ordre et post-ordre.

Exemple d'implémentation d'un arbre binaire :

\begin{lstlisting}
	typedef struct TreeNode {
		int data;
		struct TreeNode* left;
		struct TreeNode* right;
	} TreeNode;
	
	TreeNode* createNode(int data) {
		TreeNode* new_node = (TreeNode*)malloc(sizeof(TreeNode));
		new_node->data = data;
		new_node->left = NULL;
		new_node->right = NULL;
		return new_node;
	}
	
	void insert(TreeNode** root, int data) {
		if (*root == NULL) {
			*root = createNode(data);
		} else {
			if (data < (*root)->data) {
				insert(&(*root)->left, data);
			} else {
				insert(&(*root)->right, data);
			}
		}
	}
\end{lstlisting}

\section{Arbres}
Les arbres sont des structures de données génériques qui peuvent avoir plusieurs enfants. Ils sont utilisés dans des contextes comme les arbres de syntaxe abstraite, les arbres de préfixes, et les arbres de décision.

Exemple d'utilisation d'un arbre pour un arbre de décision :

\begin{lstlisting}
	typedef struct DecisionNode {
		char* question;
		struct DecisionNode* yes;
		struct DecisionNode* no;
	} DecisionNode;
	
	DecisionNode* createDecisionNode(char* question) {
		DecisionNode* node = (DecisionNode*)malloc(sizeof(DecisionNode));
		node->question = question;
		node->yes = NULL;
		node->no = NULL;
		return node;
	}
	
	void addDecision(DecisionNode* root, char* question, int answer) {
		if (answer == 1) {
			root->yes = createDecisionNode(question);
		} else {
			root->no = createDecisionNode(question);
		}
	}
\end{lstlisting}

\section{Graphes}
Les graphes sont des structures de données qui contiennent des nœuds (ou sommets) et des arêtes (qui relient les nœuds). Les graphes sont utilisés pour des problèmes complexes comme la recherche de plus courts chemins, la détection de cycles, et le partitionnement.

Exemple d'implémentation d'un graphe avec des listes d'adjacence :

\begin{lstlisting}
	typedef struct GraphNode {
		int vertex;
		struct GraphNode* next;
	} GraphNode;
	
	typedef struct Graph {
		int num_vertices;
		GraphNode** adj_lists;
	} Graph;
	
	Graph* createGraph(int num_vertices) {
		Graph* graph = (Graph*)malloc(sizeof(Graph));
		graph->num_vertices = num_vertices;
		graph->adj_lists = (GraphNode**)malloc(num_vertices * sizeof(GraphNode*));
		for (int i = 0; i < num_vertices; i++) {
			graph->adj_lists[i] = NULL;
		}
		return graph;
	}
	
	void addEdge(Graph* graph, int src, int dest) {
		GraphNode* new_node = (GraphNode*)malloc(sizeof(GraphNode));
		new_node->vertex = src;
		new_node->next = graph->adj_lists[dest];
		graph->adj_lists[dest] = new_node;
		
		new_node = (GraphNode*)malloc(sizeof(GraphNode));
		new_node->vertex = dest;
		new_node->next = graph's adj_lists[src];
		graph's adj_lists[src] = new_node;
	}
\end{lstlisting}

\section{Exercices}
Voici quelques exercices pour vous aider à pratiquer et consolider vos connaissances des structures de données avancées :

\begin{enumerate}
	\item Implémentez une liste chaînée avec des opérations d'insertion et de suppression. Testez votre implémentation avec plusieurs cas d'utilisation.
	
	\item Créez un algorithme qui utilise une pile pour évaluer une expression mathématique simple, comme "3 + 5 * 2".
	
	\item Implémentez une file qui suit le principe FIFO (First In, First Out) avec des opérations d'enfilage et de défilage. Testez-la avec différents types de données.
	
	\item Créez un arbre binaire qui peut insérer des valeurs en maintenant une structure de recherche binaire. Ajoutez des fonctions de recherche, de suppression, et de parcours.
	
	\item Implémentez un graphe qui utilise des listes d'adjacence pour représenter des connexions entre des sommets. Ajoutez des fonctions pour ajouter et supprimer des arêtes, et pour trouver des chemins entre des sommets.
\end{enumerate}
 % Chapitre 4
	\chapter{Applications d'algorithmes en IA}
%\section{Greedy algorithm}
[TO BE COMPLETED] % Chapitre 5
	\chapter{Corrig\'ees des exercices}
\section{Structures de données fondamentales}

\subsection{Variables et types de données}
\begin{enumerate}
	\item Quelle est la différence entre les types de données suivants :
	\begin{enumerate}
		\item \textbf{int}
		\item \textbf{float}
		\item \textbf{char}
	\end{enumerate}
	
	\item Écrivez un programme C qui déclare des variables pour chaque type de données mentionné et affiche leurs valeurs.
	
	 \textbf{Réponse:}
	 \begin{lstlisting}
		 #include <stdio.h>
		
		 int main() {
			     int entier = 42; 
			     float flottant = 3.14; 
			     char caractere = 'A'; 
			
			     printf("Entier: %d\n", entier);
			     printf("Flottant: %.2f\n", flottant);
			     printf("Caractere: %c\n", caractere);
			
			     return 0;
			 }
		 \end{lstlisting}
	
	\item Écrivez un programme C qui utilise des variables de différents types de données pour calculer la somme d'un entier et d'un flottant, puis affichez le résultat.
	
	 \textbf{Réponse:}
	 \begin{lstlisting}
		 #include <stdio.h>
		
		 int main() {
			     int entier = 10;
			     float flottant = 2.5;
			
			     float somme = entier + flottant; 
			
			     printf("La somme est: %.2f\n", somme);
			
			     return 0;
			 }
		 \end{lstlisting}
	
	\item Quelles sont les limites de valeurs pour les types de données suivants :
	\begin{enumerate}
		\item \textbf{int}
		\item \textbf{float}
		\item \textbf{double}
	\end{enumerate}
	\textit{Indication:} Consid\'erer le nombre de bits et octets. Par exemple, un caract\`re ASCII est repr\'esent\'e par un octet, c'est-\`a-dire 8 bits. Le caract\`ere "A" est repr\'esent\'e par 01000001. Avec huit chiffres de 0 et 1, on peut representer $2^8 = 256$ valeurs.  Pour les nombres  sign\'es (int, float, double, ...), l'oppos\'e est represent\'e diff\'eremment. 
	
	 \textbf{Réponse:}
	 \begin{enumerate}
	 	\item `int`: généralement de `-2.147.483.648` à `2.147.483.647`.
	 \item `float`: de `1,2E-38` à `3,4E+38`.
	 \item `double`: de `2,3E-308` à `1,7E+308`.
	 \end{enumerate}
\end{enumerate}

\subsection{Pointeurs}
\begin{enumerate}
	\item Quelle est la différence entre une variable et un pointeur ?
	\item Écrivez un programme C qui utilise des pointeurs pour modifier la valeur d'une variable.
	
	 \textbf{Réponse:}
	 \begin{lstlisting}
		 #include <stdio.h>
		
		 int main() {
			     int x = 10;
			     int *p = &x; 
			
			     printf("Valeur_de_x_avant_modification: %d\n", x);
			
			     *p = 20;
			
			     printf("Valeur_de_x_apres_modification: %d\n", x);
			
			     return 0;
			 }
		 \end{lstlisting}
	
	\item Écrivez un programme C qui utilise des pointeurs pour échanger les valeurs de deux variables.
	
	 \textbf{Réponse:}
	 \begin{lstlisting}
		 #include <stdio.h>
		
		 void echanger(int *a, int *b) {
			     int temp = *a;
			     *a = *b;
			     *b = temp;
			 }
		
		 int main() {
			     int x = 10;
			     int y = 20;
			
			     printf("Avant l'echange: x = %d, y = %d\n", x, y);
			
			     echanger(&x, &y);
			
			     printf("Apres l'echange: x = %d, y = %d\n", x, y);
			
			     return 0;
			 }
		 \end{lstlisting}
	
	\item Expliquez ce qu'est un pointeur nul (null pointer) et ce qui se passe si vous tentez d'utiliser un pointeur nul.
	
	 \textbf{Réponse:} Un pointeur nul (`NULL`) ne pointe pas vers une adresse mémoire valide. Tenter d'utiliser ou de déréférencer un pointeur nul peut provoquer des erreurs de segmentation ou des plantages du programme.
\end{enumerate}

\subsection{Tableaux}
\begin{enumerate}
	\item Déclarez un tableau de 5 entiers et initialisez-le avec des valeurs. Écrivez un programme C pour afficher les valeurs du tableau.
	\item Modifiez le programme pour inverser les éléments du tableau et afficher le tableau inversé.

	 \textbf{Réponse:}
	 \begin{lstlisting}
		 #include <stdio.h>
		
		 int main() {
			     int tableau[5] = {1, 2, 3, 4, 5};
			
			     printf("Tableau initial: ");
			     for (int i = 0; i < 5; i++) {
				         printf("%d ", tableau[i]);
				     }
			     printf("\n");
			
			     // Inverser le tableau
			     int temp;
			     for (int i = 0; i < 2; i++) {
				         temp = tableau[i];
				         tableau[i] = tableau[4 - i];
				         tableau[4 - i] = temp;
				     }
			
			     printf("Tableau inversE: ");
			     for (int i = 0; i < 5; i++) {
				         printf("%d ", tableau[i]);
				     }
			     printf("\n");
			
			     return 0;
			 }
		 \end{lstlisting}
	
	\item Écrivez un programme C qui calcule la somme de tous les éléments d'un tableau d'entiers de longueur `n`.


	 \textbf{Réponse:}
	 \begin{lstlisting}
		 #include <stdio.h>
		
		 int somme(int tableau[], int taille) {
			     int total = 0;
			
			     for (int i = 0; i < taille; i++) {
				         total += tableau[i];
				     }
			
			     return total;
			 }
		
		 int main() {
			     int tableau[5] = {1, 2, 3, 4, 5};
			
			     int total = somme(tableau, 5);
			
			     printf("La somme totale est: %d\n", total);
			
			     return 0;
			 }
		 \end{lstlisting}
	
	\item Que se passe-t-il si vous essayez d'accéder à un index hors limites (out of bounds) dans un tableau C ? Écrivez un programme pour le démontrer.
	
	 \textbf{Réponse:} Accéder à un index hors limites (out of bounds) peut provoquer des erreurs de segmentation ou des plantages du programme.
	 \begin{lstlisting}
		 #include <stdio.h>
		
		 int main() {
			     int tableau[5] = {1, 2, 3, 4, 5};
			
			     // Essayer d'acceder a un index hors limites
			     printf("AccEs a un index hors limites: %d\n", tableau[10]); 
			     // Ceci peut provoquer des erreurs de segmentation
			
			     return 0;
			 }
		 \end{lstlisting}
\end{enumerate}

\subsection{Chaînes de caractères}
\begin{enumerate}
	\item Quelle est la particularité des chaînes de caractères en C par rapport aux autres types de données ?
	\item Écrivez un programme C qui déclare une chaîne de caractères et utilise la bibliothèque `string.h` pour copier du texte dans la chaîne.
	
	 \textbf{Réponse:}
	 
	 \begin{lstlisting}
		 #include <stdio.h>
		 #include <string.h>
		
		 int main() {
			     char chaine[50];
			
			     strcpy(chaine, "Bonjour, monde!");
			
			     printf("Chaîne: %s\n", chaine);
			
			     return 0;
			 }
		 \end{lstlisting}
	
	\item Écrivez un programme C qui concatène deux chaînes de caractères en utilisant `strcat`.
	
	 \textbf{Réponse:}
	 \begin{lstlisting}
		 #include <stdio.h>
		 #include <string.h>
		
		 int main() {
			     char chaine1[50] = "Bonjour";
			     char chaine2[50] = ", monde!";
			
			     strcat(chaine1, chaine2); // Concaténer les deux chaînes
			
			     printf("Résultat de la concaténation: %s\n", chaine1);
			
			     return 0;
			 }
 \end{lstlisting}
	
	\item Comment compare-t-on deux chaînes de caractères en C ? Écrivez un programme qui compare deux chaînes et indique si elles sont identiques ou non.

	 \textbf{Réponse:}
	 On peut comparer deux chaînes de caractères en utilisant `strcmp` de la bibliothèque `string.h`.
	 \begin{lstlisting}
		 #include <stdio.h>
		 #include <string.h>
		
		 int main() {
			     char chaine1[50] = "Bonjour";
			     char chaine2[50] = "Bonjour";
			
			     int comparaison = strcmp(chaine1, chaine2);
			
			     if (comparaison == 0) {
				         printf("Les deux chaînes sont identiques.\n");
				     } else if (comparaison < 0) {
				         printf("La première chaîne est inférieure.\n");
				     } else {
				         printf("La première chaîne est supérieure.\n");
				     }
			
			     return 0;
			 }
		 \end{lstlisting}
\end{enumerate}

\subsection{Structures}
\begin{enumerate}
	\item Déclarez une structure en C pour représenter une personne avec un nom, un âge, et une taille.
	\item Écrivez un programme C qui utilise cette structure pour créer une personne, assigner des valeurs, et afficher les informations de la personne.
	
	 \textbf{Réponse:}
	 \begin{lstlisting}
		 #include <stdio.h>
		 #include <string.h>
		
		 typedef struct {
			     char nom[50];
			     int age;
			     float taille;
			 } Personne;
		
		 int main() {
			     Personne p;
			     strcpy(p.nom, "John Doe");
			     p.age = 30;
			     p.taille = 1.75;
			
			     printf("Nom: %s\n", p.nom);
			     printf("Âge: %d\n", p.age);
			     printf("Taille: %.2f m\n", p.taille);
			
			     return 0;
			 }
		 \end{lstlisting}
	
	\item Écrivez un programme C qui utilise des structures pour représenter un point 3D avec des coordonnées x, y, z. Ensuite, écrivez une fonction qui calcule la distance entre deux points 3D.
	
	 \textbf{Réponse:}
	 \begin{lstlisting}
		 #include <stdio.h>
		 #include <math.h>
		
		 typedef struct {
			     float x;
			     float y;
			     float z;
			 } Point3D;
		
		 float distance(Point3D p1, Point3D p2) {
			     return sqrt(pow(p2.x - p1.x, 2) + pow(p2.y - p1.y, 2) + pow(p2.z - p1.z, 2));
			 }
		
		 int main() {
			     Point3D p1 = {0, 0, 0};
			     Point3D p2 = {3, 4, 5};
			
			     float dist = distance(p1, p2);
			
			     printf("Distance entre les deux points: %.2f\n", dist);
			
			     return 0;
			 }
		 \end{lstlisting}
	
	\item Créez une structure pour représenter une voiture avec un nom, une année, et un prix. Écrivez un programme C qui utilise cette structure pour créer une voiture et afficher ses propriétés.

	 \textbf{Réponse:}
	 \begin{lstlisting}
		 #include <stdio.h>
		 #include <string.h>
		
		 typedef struct {
			     char nom[50];
			     int annee;
			     float prix;
			 } Voiture;
		
		 int main() {
			     Voiture v;
			     strcpy(v.nom, "Toyota");
			     v.annee = 2020;
			     v.prix = 25000.50;
			
			     printf("Voiture: %s\n", v.nom);
			     printf("Année: %d\n", v.annee);
			     printf("Prix: %.2f\n", v.prix);
			
			     return 0;
			 }
		 \end{lstlisting}
\end{enumerate}

\section{Algorithmes de tri et de recherche}

\subsection{Algorithmes de tri}

\begin{enumerate}
	\item Pour chaque algorithme de tri, implémentez-le en C et testez-le avec les cas suivants :
	\begin{enumerate}
		\item Un tableau vide.
		\item Un tableau contenant un seul élément.
		\item Un tableau déjà trié.
		\item Le tableau \([9, 1, 9, 0, 2, 6, 6, 5, 3, 4, 7]\).
	\end{enumerate}
	
	\subsubsection{Tri à bulle (Bubble Sort)}
	
	\textbf{Réponse:}
	\begin{lstlisting}
		#include <stdio.h>
		
		void bubbleSort(int arr[], int n) {
			for (int i = 0; i < n-1; i++) {
				for (int j = 0; j < n-i-1; j++) {
					if (arr[j] > arr[j+1]) {
						// Échanger les éléments
						int temp = arr[j];
						arr[j] = arr[j+1];
						arr[j+1] = temp;
					}
				}
			}
		}
		
		void printArray(int arr[], int size) {
			for (int i = 0; i < size; i++) {
				printf("%d ", arr[i]);
			}
			printf("\n");
		}
		
		int main() {
			int arr[] = {9, 1, 9, 0, 2, 6, 6, 5, 3, 4, 7};
			int n = sizeof(arr)/sizeof(arr[0]);
			
			bubbleSort(arr, n);
			printf("Array trié : ");
			printArray(arr, n);
			
			return 0;
		}
	\end{lstlisting}
	
	\subsubsection{Tri par insertion (Insertion Sort)}
	
	\textbf{Réponse:}
	\begin{lstlisting}
		#include <stdio.h>
		
		void insertionSort(int arr[], int n) {
			for (int i = 1; i < n; i++) {
				int key = arr[i];
				int j = i - 1;
				while (j >= 0 && arr[j] > key) {
					arr[j + 1] = arr[j];
					j = j - 1;
				}
				arr[j + 1] = key;
			}
		}
		
		void printArray(int arr[], int size) {
			for (int i = 0; i < size; i++) {
				printf("%d ", arr[i]);
			}
			printf("\n");
		}
		
		int main() {
			int arr[] = {9, 1, 9, 0, 2, 6, 6, 5, 3, 4, 7};
			int n = sizeof(arr)/sizeof(arr[0]);
			
			insertionSort(arr, n);
			printf("Array trié : ");
			printArray(arr, n);
			
			return 0;
		}
	\end{lstlisting}
	
	\subsubsection{Tri rapide (Quick Sort)}
	
	\textbf{Réponse:}
	\begin{lstlisting}
		#include <stdio.h>
		
		int partition(int arr[], int low, int high) {
			int pivot = arr[high];
			int i = (low - 1);
			
			for (int j = low; j < high; j++) {
				if (arr[j] < pivot) {
					i++;
					int temp = arr[i];
					arr[i] = arr[j];
					arr[j] = temp;
				}
			}
			int temp = arr[i + 1];
			arr[i + 1] = arr[high];
			arr[high] = temp;
			return i + 1;
		}
		
		void quickSort(int arr[], int low, int high) {
			if (low < high) {
				int pi = partition(arr, low, high);
				quickSort(arr, low, pi - 1);
				quickSort(arr, pi + 1, high);
			}
		}
		
		void printArray(int arr[], int size) {
			for (int i = 0; i < size; i++) {
				printf("%d ", arr[i]);
			}
			printf("\n");
		}
		
		int main() {
			int arr[] = {9, 1, 9, 0, 2, 6, 6, 5, 3, 4, 7};
			int n = sizeof(arr)/sizeof(arr[0]);
			
			quickSort(arr, 0, n - 1);
			printf("Array trié : ");
			printArray(arr, n);
			
			return 0;
		}
	\end{lstlisting}
\end{enumerate}

\subsection{Algorithmes de recherche}

\begin{enumerate}
	\item Pour chaque algorithme de recherche, implémentez-le en C et testez-le en cherchant les éléments 8, 1 et 0 dans le tableau \([9, 1, 9, 0, 2, 6, 6, 5, 3, 4, 7]\).
	
	\subsubsection{Recherche linéaire (Linear Search)}
	
	\textbf{Réponse:}
	\begin{lstlisting}
		#include <stdio.h>
		
		int linearSearch(int arr[], int n, int key) {
			for (int i = 0; i < n; i++) {
				if (arr[i] == key) {
					return i; // Retourne l'indice où l'élément est trouvé
				}
			}
			return -1; // L'élément n'est pas trouvé
		}
		
		int main() {
			int arr[] = {9, 1, 9, 0, 2, 6, 6, 5, 3, 4, 7};
			int n = sizeof(arr)/sizeof(arr[0]);
			
			printf("Recherche de l'élément 8: %d\n", linearSearch(arr, n, 8));
			printf("Recherche de l'élément 1: %d\n", linearSearch(arr, n, 1));
			printf("Recherche de l'élément 0: %d\n", linearSearch(arr, n, 0));
			
			return 0;
		}
	\end{lstlisting}
	
	\subsubsection{Recherche binaire (Binary Search)}
	
	La recherche binaire n'est applicable que si le tableau est trié. Voici l'implémentation pour une recherche binaire dans un tableau trié.
	
	\textbf{Réponse:}
	\begin{lstlisting}
		#include <stdio.h>
		
		int binarySearch(int arr[], int low, int high, int key) {
			while (low <= high) {
				int mid = low + (high - low) / 2;
				if (arr[mid] == key) {
					return mid;
				}
				if (arr[mid] < key) {
					low = mid + 1;
				} else {
					high = mid - 1;
				}
			}
			return -1; // L'élément n'est pas trouvé
		}
		
		int main() {
			int arr[] = {0, 1, 4, 4, 4, 5, 6, 7, 9, 9};
			int n = sizeof(arr)/sizeof(arr[0]);
			
			printf("Recherche de l'élément 0: %d\n", binarySearch(arr, 0, n - 1, 0));
			printf("Recherche de l'élément 2: %d\n", binarySearch(arr, 0, n - 1, 2));
			printf("Recherche de l'élément 7: %d\n", binarySearch(arr, 0, n - 1, 7));
			printf("Recherche de l'élément 9: %d\n", binarySearch(arr, 0, n - 1, 9));
			
			return 0;
		}
	\end{lstlisting}
	
	\item Lequel de ces algorithmes n'est pas applicable au tableau précédent ? Pourquoi ?
	
	La recherche binaire n'est pas applicable au tableau \([9, 1, 9, 0, 2, 6, 6, 5, 3, 4, 7]\) car il n'est pas trié. La recherche binaire fonctionne uniquement sur des tableaux triés, car elle divise le tableau en deux parties et suppose que les éléments à gauche sont inférieurs à ceux à droite. Dans ce cas, un tri préalable est nécessaire.
\end{enumerate}

\section{Complexité algorithmique}

%\section{Exercices}

\begin{enumerate}
	\item Quelle est la complexité temporelle des opérations suivantes ?
	\begin{enumerate}
		\item Parcourir un tableau d'entiers de longueur $n$.
		
		\textbf{Réponse} : 
		La complexité temporelle de l'opération "parcourir un tableau d'entiers" est $O(n)$. En effet, dans cette opération, chaque élément du tableau est visité une fois, ce qui signifie qu'il y a $n$ opérations à effectuer, où $n$ est la taille du tableau. La notation $O(n)$ exprime que la durée de l'opération est linéaire par rapport à la taille du tableau.
		
		\item Multiplier deux matrices de taille $n \times n$.
		
		\textbf{Réponse} : 
		La multiplication de deux matrices carrées de taille $n \times n$ nécessite d'effectuer $n^2$ multiplications. En effet, chaque élément de la matrice résultat est le produit de $n$ éléments (lignes de la première matrice et colonnes de la deuxième matrice). Donc, pour chaque élément de la matrice résultat (qui est $n^2$ éléments au total), nous devons effectuer $n$ multiplications, ce qui donne une complexité de $O(n^3)$. 
		Cette complexité est dans le cas général pour une multiplication de matrices à l'aide de l'algorithme classique. Des algorithmes plus avancés, comme l'algorithme de Strassen, peuvent réduire cette complexité à $O(n^{2.81})$, mais l'algorithme classique reste de complexité $O(n^3)$.
		
		\item Effectuer une recherche binaire sur un tableau trié.
		
		\textbf{Réponse} : 
		La recherche binaire dans un tableau trié a une complexité temporelle de $O(\log n)$. L'idée principale de la recherche binaire est de diviser à chaque étape le tableau en deux sous-tableaux, réduisant ainsi de moitié l'espace de recherche à chaque itération. Cette opération est répétée jusqu'à ce que l'élément recherché soit trouvé ou que l'espace de recherche soit vide. 
		La complexité $\log n$ provient du fait que l'on réduit par moitié la taille de la partie restante du tableau à chaque itération. La recherche binaire est donc très efficace pour les tableaux triés, car elle évite de parcourir l'intégralité du tableau. Contrairement à une recherche linéaire qui aurait une complexité $O(n)$, la recherche binaire est logarithmique.
		
	\end{enumerate}
	
	\item Quelle est la complexité temporelle du code suivant ?
	\begin{lstlisting}
		for (int i = 0; i < n; i++) {
			for (int j = 0; j < n; j++) {
				printf("%d ", i * j);
			}
		}
	\end{lstlisting}
	
	\textbf{Réponse} :
	La complexité temporelle de ce code est $O(n^2)$. Voici pourquoi :
	
	- La première boucle \texttt{for (int i = 0; i < n; i++)} s'exécute $n$ fois, car $i$ varie de $0$ à $n-1$.
	- La deuxième boucle imbriquée \texttt{for (int j = 0; j < n; j++)} s'exécute également $n$ fois pour chaque itération de la première boucle.
	
	Par conséquent, pour chaque itération de la première boucle (qui se répète $n$ fois), la deuxième boucle s'exécute également $n$ fois, ce qui donne un total de $n \times n = n^2$ itérations au total. C'est ce que l'on appelle une complexité quadratique.
	
	- Le \texttt{printf} à l'intérieur des boucles effectue une opération constante à chaque itération (une simple impression), ce qui ne change pas la complexité globale de l'algorithme. L'opération $\texttt{printf}$ elle-même, bien que coûteuse en termes de temps d'exécution réel, reste considérée comme une opération de complexité constante dans ce contexte.
	
	\item Comment pouvez-vous améliorer cet algorithme pour réduire sa complexité ?
	
	\textbf{Réponse} :
	Pour réduire la complexité de cet algorithme, il faudrait d'abord identifier s'il y a une optimisation possible dans la logique de l'algorithme. Ici, le problème semble être celui d'afficher tous les produits de $i \times j$ pour $i, j$ allant de $0$ à $n-1$. 
	Une possible amélioration serait de revoir la structure du code pour éviter des calculs redondants. En effet, si on cherche à afficher les produits $i \times j$ et que l'on parcourt les deux indices $i$ et $j$ indépendamment, cela entraîne une complexité $O(n^2)$. Cependant, il est possible de réorganiser le calcul en utilisant des propriétés symétriques des produits, comme dans le cas d'un tableau triangulaire ou d'autres structures spécifiques.
	
	En fonction de l'objectif spécifique du programme (comme stocker les résultats ou effectuer un calcul particulier), il pourrait être possible de réduire la quantité de travail en évitant de calculer des produits pour des paires $(i, j)$ identiques (par exemple, en exploitant la symétrie si $i \times j = j \times i$), mais cela dépend du contexte de l'application.
\end{enumerate}

\section{Structures de données avancées}

\subsection{Listes chaînées}

\begin{enumerate}
	\item \textbf{Création d'une liste chaînée} :
	
	\textbf{Réponse} :
	Pour créer une liste chaînée, nous devons définir une structure pour les nœuds de la liste, où chaque nœud contiendra une donnée et un pointeur vers le nœud suivant. La tête de la liste est initialisée à \texttt{NULL} pour indiquer que la liste est vide.
	
	\begin{lstlisting}[language=C]
		#include <stdio.h>
		#include <stdlib.h>
		
		struct Node {
			int data;
			struct Node* next;
		};
		
		struct Node* head = NULL;  // Initialisation de la tête de la liste à NULL
		
		// Fonction pour créer une nouvelle liste
		void createList(int value) {
			struct Node* newNode = malloc(sizeof(struct Node));
			newNode->data = value;
			newNode->next = NULL;
			head = newNode;
		}
	\end{lstlisting}
	
	\textbf{Explication} :
	La fonction \texttt{createList} alloue de la mémoire pour un nouveau nœud, définit sa donnée, et initialise son pointeur suivant à \texttt{NULL}. Elle affecte ensuite ce nœud à la tête de la liste.
	
	\item \textbf{Insertion d'un nœud au début de la liste} :
	
	\textbf{Réponse} :
	Pour insérer un nœud au début de la liste chaînée, nous créons un nouveau nœud et nous mettons à jour son pointeur \texttt{next} pour qu'il pointe vers l'ancienne tête de la liste. Ensuite, nous mettons la tête de la liste pour qu'elle pointe vers ce nouveau nœud.
	
	\begin{lstlisting}[language=C]
		// Fonction pour insérer un nœud au début de la liste
		void insertAtBeginning(int value) {
			struct Node* newNode = malloc(sizeof(struct Node));
			newNode->data = value;
			newNode->next = head;  // Le nouveau nœud pointe vers l'ancienne tête
			head = newNode;  // La tête pointe maintenant vers le nouveau nœud
		}
	\end{lstlisting}
	
	\textbf{Explication} :
	La fonction \texttt{insertAtBeginning} crée un nouveau nœud avec la valeur donnée, fait pointer ce nœud vers l'ancienne tête de la liste, puis met la tête de la liste pour qu'elle pointe vers le nouveau nœud. Cela insère efficacement le nœud au début de la liste.
	
	\item \textbf{Insertion d'un nœud à une position spécifique} :
	
	\textbf{Réponse} :
	Pour insérer un nœud à une position spécifique dans la liste, nous devons parcourir la liste jusqu'à la position souhaitée, puis insérer le nœud à cet endroit en ajustant les pointeurs. Si la position est zéro, on insère au début de la liste.
	
	\begin{lstlisting}[language=C]
		// Fonction pour insérer un nœud à une position spécifique
		void insertAtPosition(int value, int position) {
			if (position == 0) {
				insertAtBeginning(value);
				return;
			}
			
			struct Node* newNode = malloc(sizeof(struct Node));
			newNode->data = value;
			struct Node* current = head;
			
			for (int i = 0; i < position - 1 && current != NULL; i++) {
				current = current->next;
			}
			
			if (current == NULL) {
				printf("Position invalide\n");
				return;
			}
			
			newNode->next = current->next;
			current->next = newNode;
		}
	\end{lstlisting}
	
	\textbf{Explication} :
	La fonction \texttt{insertAtPosition} parcourt la liste jusqu'à la position spécifiée. Si la position est zéro, elle appelle la fonction \texttt{insertAtBeginning}. Sinon, elle insère un nouveau nœud à la position donnée en ajustant les pointeurs. Si la position est invalide, un message d'erreur est affiché.
	
	\item \textbf{Suppression d'un nœud de la liste chaînée} :
	
	\textbf{Réponse} :
	Pour supprimer un nœud de la liste chaînée, nous devons parcourir la liste jusqu'à ce que nous trouvions le nœud à supprimer. Ensuite, nous ajustons le pointeur du nœud précédent pour qu'il pointe vers le nœud suivant du nœud à supprimer.
	
	\begin{lstlisting}[language=C]
		// Fonction pour supprimer un nœud d'une liste chaînée
		void deleteNode(int value) {
			struct Node* current = head;
			struct Node* previous = NULL;
			
			// Si la tête contient la valeur à supprimer
			if (current != NULL && current->data == value) {
				head = current->next;
				free(current);
				return;
			}
			
			// Parcours de la liste à la recherche du nœud
			while (current != NULL && current->data != value) {
				previous = current;
				current = current->next;
			}
			
			// Si la valeur n'est pas trouvée
			if (current == NULL) {
				printf("Valeur non trouvée dans la liste\n");
				return;
			}
			
			// Supprimer le nœud
			previous->next = current->next;
			free(current);
		}
	\end{lstlisting}
	
	\textbf{Explication} :
	La fonction \texttt{deleteNode} commence par vérifier si le nœud à supprimer est le premier nœud de la liste (la tête). Si ce n'est pas le cas, elle parcourt la liste pour trouver le nœud correspondant à la valeur à supprimer. Une fois le nœud trouvé, elle ajustera le pointeur du nœud précédent pour qu'il saute le nœud à supprimer et libère la mémoire allouée pour ce nœud.
	
\end{enumerate}




\subsection{Piles}

\begin{enumerate}
	\item \textbf{Construction d'une pile} :
	
	\textbf{Réponse} :
	Pour construire une pile, vous devez définir une structure qui contient les éléments de la pile et un pointeur ou un indice qui indique le sommet de la pile. La structure suivante utilise un tableau pour représenter la pile et un indice pour suivre le sommet de la pile.
	
	\begin{lstlisting}[language=C]
		#include <stdio.h>
		#include <stdlib.h>
		
		#define MAX 10
		
		struct Stack {
			int items[MAX];
			int top;
		};
		
		// Fonction pour initialiser la pile
		void initStack(struct Stack* stack) {
			stack->top = -1;  // La pile est vide au départ
		}
		
		// Fonction pour vérifier si la pile est vide
		int isEmpty(struct Stack* stack) {
			return stack->top == -1;
		}
	\end{lstlisting}
	
	\textbf{Explication} :
	La fonction \texttt{initStack} initialise la pile en définissant le sommet à \texttt{-1}, ce qui indique que la pile est vide. La fonction \texttt{isEmpty} permet de vérifier si la pile est vide en vérifiant si l'indice du sommet est égal à \texttt{-1}.
	
	\item \textbf{Insertion d'un élément dans une pile (empilage)} :
	
	\textbf{Réponse} :
	L'empilage consiste à ajouter un élément au sommet de la pile. Voici la fonction pour empiler un élément dans la pile, en vérifiant d'abord si la pile n'est pas pleine.
	
	\begin{lstlisting}[language=C]
		// Fonction pour empiler un élément
		void push(struct Stack* stack, int value) {
			if (stack->top == MAX - 1) {
				printf("Erreur : pile pleine\n");
				return;
			}
			stack->items[++stack->top] = value;  // Incrémente le sommet puis ajoute l'élément
		}
	\end{lstlisting}
	
	\textbf{Explication} :
	La fonction \texttt{push} ajoute un élément au sommet de la pile. Si le sommet est égal à \texttt{MAX - 1}, cela signifie que la pile est pleine et une erreur est renvoyée. Sinon, l'élément est ajouté au sommet, et l'indice du sommet est incrémenté.
	
	\item \textbf{Suppression d'un élément dans une pile (dépilage)} :
	
	\textbf{Réponse} :
	Le dépilage consiste à retirer l'élément du sommet de la pile. Si la pile est vide, une erreur est renvoyée. Voici la fonction pour dépiler un élément de la pile :
	
	\begin{lstlisting}[language=C]
		// Fonction pour dépiler un élément
		int pop(struct Stack* stack) {
			if (isEmpty(stack)) {
				printf("Erreur : pile vide\n");
				return -1;  // Valeur spéciale pour indiquer une erreur
			}
			return stack->items[stack->top--];  // Retourne l'élément au sommet puis décrémente le sommet
		}
	\end{lstlisting}
	
	\textbf{Explication} :
	La fonction \texttt{pop} supprime l'élément au sommet de la pile. Si la pile est vide, elle renvoie une erreur. Sinon, elle retourne l'élément au sommet et décrémente l'indice du sommet pour retirer l'élément de la pile.
	
	\item \textbf{Test de la pile} :
	
	\textbf{Réponse} :
	Voici un exemple de code qui utilise les fonctions d'empilage et de dépilage sur une pile.
	
	\begin{lstlisting}[language=C]
		int main() {
			struct Stack stack;
			initStack(&stack);
			
			push(&stack, 10);
			push(&stack, 20);
			push(&stack, 30);
			
			printf("Dépiler : %d\n", pop(&stack));  // Devrait afficher 30
			printf("Dépiler : %d\n", pop(&stack));  // Devrait afficher 20
			printf("Dépiler : %d\n", pop(&stack));  // Devrait afficher 10
			
			return 0;
		}
	\end{lstlisting}
	
	\textbf{Explication} :
	Dans cet exemple, nous empilons trois éléments (10, 20, 30) dans la pile et ensuite nous les dépilons un par un. Les éléments sont retirés dans l'ordre inverse de leur insertion, conformément au principe LIFO (Last In, First Out).
	
\end{enumerate}

%

\subsection{Files}

\begin{enumerate}
	\item \textbf{Construction d'une file} :
	
	\textbf{Réponse} :
	Pour construire une file, vous devez définir une structure qui contient les éléments de la file, ainsi que des pointeurs pour l'avant et l'arrière. Ces pointeurs permettent de gérer les opérations d'enfilage (ajout d'un élément à l'arrière) et de défilage (retrait du premier élément). Voici un exemple de structure de file en C :
	
	\begin{lstlisting}[language=C]
		#include <stdio.h>
		#include <stdlib.h>
		
		struct Node {
			int data;
			struct Node* next;
		};
		
		struct Queue {
			struct Node* front;
			struct Node* rear;
		};
		
		// Fonction pour initialiser la file
		void initQueue(struct Queue* q) {
			q->front = NULL;
			q->rear = NULL;
		}
	\end{lstlisting}
	
	\textbf{Explication} :
	La structure \texttt{Queue} contient deux pointeurs, \texttt{front} et \texttt{rear}, qui pointent respectivement vers le premier et le dernier élément de la file. La fonction \texttt{initQueue} initialise ces pointeurs à \texttt{NULL} pour indiquer que la file est vide.
	
	\item \textbf{Insertion d'un élément dans une file (enfilage)} :
	
	\textbf{Réponse} :
	L'enfilage consiste à ajouter un élément à la fin de la file. Si la file est vide, l'élément ajouté devient à la fois l'avant et l'arrière de la file. Si la file contient déjà des éléments, vous devez ajuster le pointeur \texttt{rear} pour indiquer le nouvel élément.
	
	\begin{lstlisting}[language=C]
		// Fonction pour enfiler un élément
		void enqueue(struct Queue* q, int value) {
			struct Node* newNode = malloc(sizeof(struct Node));
			newNode->data = value;
			newNode->next = NULL;
			
			if (q->rear == NULL) {  // Si la file est vide
				q->front = newNode;
				q->rear = newNode;
			} else {
				q->rear->next = newNode;
				q->rear = newNode;
			}
		}
	\end{lstlisting}
	
	\textbf{Explication} :
	La fonction \texttt{enqueue} crée un nouveau nœud et le place à la fin de la file. Si la file est vide, l'élément ajouté devient à la fois le premier et le dernier élément. Sinon, le nouvel élément devient le dernier de la file, et \texttt{rear} est mis à jour pour pointer vers ce nouvel élément.
	
	\item \textbf{Suppression d'un élément dans une file (défilage)} :
	
	\textbf{Réponse} :
	Le défilage consiste à retirer le premier élément de la file. Si la file est vide, la fonction retourne une erreur ou une valeur spéciale. Sinon, elle met à jour le pointeur \texttt{front} pour qu'il pointe vers le prochain élément de la file et libère la mémoire du nœud retiré.
	
	\begin{lstlisting}[language=C]
		// Fonction pour défiler un élément
		int dequeue(struct Queue* q) {
			if (q->front == NULL) {
				printf("La file est vide!\n");
				return -1;  // Valeur d'erreur
			}
			
			struct Node* temp = q->front;
			int value = temp->data;
			q->front = q->front->next;
			
			if (q->front == NULL) {  // Si la file devient vide après le défilage
				q->rear = NULL;
			}
			
			free(temp);
			return value;
		}
	\end{lstlisting}
	
	\textbf{Explication} :
	La fonction \texttt{dequeue} retire l'élément au début de la file en mettant à jour le pointeur \texttt{front}. Si la file devient vide après le défilage, le pointeur \texttt{rear} est également mis à \texttt{NULL}. La mémoire du nœud retiré est libérée, et la fonction retourne la valeur de l'élément retiré.
	
\end{enumerate}

%

\subsection{Arbres binaires}

\begin{enumerate}
	\item \textbf{Construction d'un arbre binaire} :
	
	\textbf{Réponse} :
	Pour construire un arbre binaire, vous devez définir une structure qui représente chaque nœud de l'arbre. Chaque nœud contient des données et deux pointeurs, un pour l'enfant gauche et un pour l'enfant droit. L'arbre commence avec un seul nœud racine, et les autres nœuds sont ajoutés en fonction de leur relation avec la racine. Voici un exemple de code en C pour définir la structure d'un arbre binaire :
	
	\begin{lstlisting}[language=C]
		#include <stdio.h>
		#include <stdlib.h>
		
		struct Node {
			int data;
			struct Node* left;
			struct Node* right;
		};
		
		// Fonction pour créer un nouveau nœud
		struct Node* createNode(int data) {
			struct Node* newNode = malloc(sizeof(struct Node));
			newNode->data = data;
			newNode->left = NULL;
			newNode->right = NULL;
			return newNode;
		}
		
		// Fonction pour insérer un élément dans l'arbre binaire
		void insert(struct Node** root, int data) {
			if (*root == NULL) {
				*root = createNode(data);
				return;
			}
			if (data < (*root)->data)
			insert(&(*root)->left, data);
			else
			insert(&(*root)->right, data);
		}
	\end{lstlisting}
	
	\textbf{Explication} :
	La structure \texttt{Node} contient un entier \texttt{data} et deux pointeurs \texttt{left} et \texttt{right} qui pointent respectivement vers l'enfant gauche et l'enfant droit. La fonction \texttt{createNode} crée un nouveau nœud avec la valeur donnée, et la fonction \texttt{insert} insère un élément dans l'arbre en suivant la règle de l'arbre binaire de recherche (les éléments plus petits vont à gauche et les plus grands à droite).
	
	\item \textbf{Insertion dans un arbre binaire} :
	
	\textbf{Réponse} :
	L'insertion dans un arbre binaire se fait en suivant la règle de comparaison avec la racine. Si le nouvel élément est plus petit que la racine, il va à gauche, sinon il va à droite. Le processus continue récursivement jusqu'à ce qu'un emplacement vide soit trouvé pour le nouvel élément.
	
	Voici l'exemple d'insertion dans un arbre binaire :
	
	\begin{lstlisting}[language=C]
		int main() {
			struct Node* root = NULL;
			insert(&root, 50);  // Insertion de 50 à la racine
			insert(&root, 30);  // Insertion de 30 à gauche de 50
			insert(&root, 70);  // Insertion de 70 à droite de 50
			insert(&root, 20);  // Insertion de 20 à gauche de 30
			insert(&root, 40);  // Insertion de 40 à droite de 30
			insert(&root, 60);  // Insertion de 60 à gauche de 70
			insert(&root, 80);  // Insertion de 80 à droite de 70
			return 0;
		}
	\end{lstlisting}
	
	\textbf{Explication} :
	Dans cet exemple, l'arbre est construit à partir de la racine et chaque élément est inséré à la bonne position en fonction de sa valeur par rapport à la racine. Les éléments plus petits que la racine vont à gauche et ceux plus grands vont à droite. L'insertion est effectuée récursivement jusqu'à ce que l'emplacement correct soit trouvé.
	
	\item \textbf{Algorithmes de parcours des arbres binaires} :
	
	\textbf{Réponse} :
	Les algorithmes de parcours des arbres binaires sont utilisés pour visiter tous les nœuds de l'arbre dans un ordre spécifique. Il existe trois principaux types de parcours d'arbres binaires :
	
	\begin{itemize}
		\item \textbf{Parcours pré-ordre} : Visite la racine, puis le sous-arbre gauche et enfin le sous-arbre droit.
		\item \textbf{Parcours en-ordre} : Visite le sous-arbre gauche, puis la racine, et enfin le sous-arbre droit. Ce type de parcours est utilisé dans les arbres binaires de recherche pour parcourir les éléments dans l'ordre croissant.
		\item \textbf{Parcours post-ordre} : Visite le sous-arbre gauche, puis le sous-arbre droit, et enfin la racine.
	\end{itemize}
	
	Exemple de code pour un parcours en-ordre en C :
	
	\begin{lstlisting}[language=C]
		// Fonction de parcours en-ordre
		void inorder(struct Node* root) {
			if (root == NULL) {
				return;
			}
			inorder(root->left);  // Visiter le sous-arbre gauche
			printf("%d ", root->data);  // Visiter la racine
			inorder(root->right);  // Visiter le sous-arbre droit
		}
		
		// Fonction pour afficher l'arbre en-ordre
		int main() {
			struct Node* root = NULL;
			insert(&root, 50);
			insert(&root, 30);
			insert(&root, 70);
			insert(&root, 20);
			insert(&root, 40);
			insert(&root, 60);
			insert(&root, 80);
			
			printf("Parcours en-ordre : ");
			inorder(root);  // Afficher les éléments de l'arbre en-ordre
			return 0;
		}
	\end{lstlisting}
	
	\textbf{Explication} :
	La fonction \texttt{inorder} parcourt l'arbre en-ordre en visitant d'abord le sous-arbre gauche, puis la racine, puis le sous-arbre droit. Le résultat du parcours en-ordre d'un arbre binaire de recherche est toujours les éléments dans l'ordre croissant.
	
\end{enumerate}



\subsection{Graphes}

\begin{enumerate}
	\item \textbf{Construction d'un graphe} :
	
	\textbf{Réponse} :
	Pour construire un graphe, il faut définir une structure de données qui contient les sommets (ou nœuds) ainsi que les arêtes qui les relient. On peut utiliser une structure avec des listes d'adjacence ou des matrices d'adjacence pour représenter le graphe. Dans ce cas, nous allons utiliser des listes d'adjacence, où chaque sommet a une liste de connexions vers les autres sommets.
	
	Voici un exemple de code C pour définir un graphe avec des listes d'adjacence :
	
	\begin{lstlisting}[language=C]
		#include <stdio.h>
		#include <stdlib.h>
		
		// Structure représentant un nœud
		struct Node {
			int vertex;
			struct Node* next;
		};
		
		// Structure représentant un graphe
		struct Graph {
			int numVertices;
			struct Node** adjLists;
		};
		
		// Fonction pour créer un nœud
		struct Node* createNode(int vertex) {
			struct Node* newNode = malloc(sizeof(struct Node));
			newNode->vertex = vertex;
			newNode->next = NULL;
			return newNode;
		}
		
		// Fonction pour créer un graphe avec un nombre de sommets donné
		struct Graph* createGraph(int vertices) {
			struct Graph* graph = malloc(sizeof(struct Graph));
			graph->numVertices = vertices;
			graph->adjLists = malloc(vertices * sizeof(struct Node*));
			
			for (int i = 0; i < vertices; i++) {
				graph->adjLists[i] = NULL; // Initialisation de chaque liste d'adjacence
			}
			return graph;
		}
		
	\end{lstlisting}
	
	\textbf{Explication} :
	Dans ce code, nous avons une structure \texttt{Node} qui représente un nœud avec un sommet et un pointeur vers le nœud suivant (pour former une liste chaînée). La structure \texttt{Graph} contient un tableau de pointeurs vers les listes d'adjacence. Chaque élément de ce tableau représente un sommet du graphe, et chaque liste contient les sommets connectés à celui-ci.
	
	\item \textbf{Ajout d'arêtes dans un graphe} :
	
	\textbf{Réponse} :
	Pour ajouter des arêtes dans un graphe, on insère un nouveau nœud dans la liste d'adjacence du sommet de départ (source) et dans la liste d'adjacence du sommet de destination. Si le graphe est orienté, l'arête va seulement du sommet source vers le sommet destination. Si le graphe est non orienté, l'arête est bidirectionnelle, donc on doit ajouter des connexions dans les deux directions.
	
	Voici un exemple de fonction en C pour ajouter une arête dans un graphe non orienté :
	
	\begin{lstlisting}[language=C]
		// Fonction pour ajouter une arête entre deux sommets dans un graphe non orienté
		void addEdge(struct Graph* graph, int src, int dest) {
			// Ajouter une arête de src à dest
			struct Node* newNode = createNode(dest);
			newNode->next = graph->adjLists[src];
			graph->adjLists[src] = newNode;
			
			// Ajouter une arête de dest à src (pour un graphe non orienté)
			newNode = createNode(src);
			newNode->next = graph->adjLists[dest];
			graph->adjLists[dest] = newNode;
		}
		
	\end{lstlisting}
	
	\textbf{Explication} :
	La fonction \texttt{addEdge} crée un nouveau nœud pour chaque arête et l'ajoute à la liste d'adjacence correspondante. Dans le cas d'un graphe non orienté, il y a deux ajouts : un de \texttt{src} à \texttt{dest} et un de \texttt{dest} à \texttt{src}, ce qui rend l'arête bidirectionnelle.
	
	\item \textbf{Recherche en largeur (Complétez le code)} :
	
	\textbf{Réponse} :
	Pour implémenter la recherche en largeur (BFS), on utilise une file (FIFO) pour explorer chaque niveau du graphe. On commence à partir du sommet source et on explore tous ses voisins, puis les voisins de ces voisins, et ainsi de suite.
	
	Voici un exemple de code pour implémenter la recherche en largeur (BFS) :
	
	\begin{lstlisting}[language=C]
		#include <stdio.h>
		#include <stdlib.h>
		#include <stdbool.h>
		
		// Structure représentant une file
		struct Queue {
			int* items;
			int front, rear, size;
		};
		
		// Fonction pour initialiser la file
		void initQueue(struct Queue* q, int size) {
			q->items = malloc(size * sizeof(int));
			q->front = 0;
			q->rear = 0;
			q->size = size;
		}
		
		// Fonction pour ajouter un élément à la file
		void enqueue(struct Queue* q, int item) {
			q->items[q->rear++] = item;
		}
		
		// Fonction pour retirer un élément de la file
		int dequeue(struct Queue* q) {
			return q->items[q->front++];
		}
		
		// Fonction pour vérifier si la file est vide
		bool isEmpty(struct Queue* q) {
			return q->front == q->rear;
		}
		
		// Fonction pour effectuer une recherche en largeur (BFS)
		void bfs(struct Graph* graph, int start) {
			bool* visited = malloc(graph->numVertices * sizeof(bool));
			for (int i = 0; i < graph->numVertices; i++) {
				visited[i] = false;
			}
			
			struct Queue q;
			initQueue(&q, graph->numVertices);
			
			visited[start] = true;
			enqueue(&q, start);
			
			while (!isEmpty(&q)) {
				int current = dequeue(&q);
				printf("%d ", current);
				
				struct Node* temp = graph->adjLists[current];
				while (temp) {
					if (!visited[temp->vertex]) {
						visited[temp->vertex] = true;
						enqueue(&q, temp->vertex);
					}
					temp = temp->next;
				}
			}
		}
		
	\end{lstlisting}
	
	\textbf{Explication} :
	La fonction \texttt{bfs} commence par marquer le sommet source comme visité et l'ajoute à la file. Ensuite, elle explore tous les voisins du sommet actuel, en ajoutant les non visités à la file. Cela continue jusqu'à ce que tous les sommets accessibles aient été visités.
	
\end{enumerate}

\section{Applications d'algorithmes en IA}
	
\section*{Régression Linéaire}

\subsection*{Données}

Les données d'entrée sont les suivantes :

\begin{table}[h]
	\centering
	\caption{Données pour la régression linéaire}
	\begin{tabular}{|c|c|}
		\hline
		\text{Surface (m²)} & \text{Prix (MGA)} \\ \hline
		50 & 70,000,000 \\ \hline
		75 & 95,000,000 \\ \hline
		100 & 130,000,000 \\ \hline
	\end{tabular}
\end{table}

Nous allons utiliser ces données pour calculer les coefficients \(a\) et \(b\) de la régression linéaire, puis prédire \(y\) pour \(x = 80\).

\subsection*{Implémentation en C}

Voici l'implémentation en C pour résoudre l'exercice :

\begin{lstlisting}[caption=Implémentation de la Régression Linéaire en C]
	#include <stdio.h>
	
	// Fonction pour calculer les coefficients a et b
	void linearRegression(int x[], int y[], int n, float *a, float *b) {
		int sumX = 0, sumY = 0, sumXY = 0, sumX2 = 0;
		
		// Calcul des sommes nécessaires
		for (int i = 0; i < n; i++) {
			sumX += x[i];
			sumY += y[i];
			sumXY += x[i] * y[i];
			sumX2 += x[i] * x[i];
		}
		
		// Calcul des coefficients
		*a = (n * sumXY - sumX * sumY) / (float)(n * sumX2 - sumX * sumX);
		*b = (sumY - (*a) * sumX) / (float)n;
	}
	
	// Fonction pour prédire y à partir de x
	float predictY(float a, float b, int x) {
		return a * x + b;
	}
	
	int main() {
		// Données fournies
		int x[] = {50, 75, 100}; // Surface des maisons en m²
		int y[] = {70000000, 95000000, 130000000}; // Prix en MGA
		int n = 3; // Nombre de paires de données
		
		float a, b;
		
		// Appel de la fonction de régression linéaire
		linearRegression(x, y, n, &a, &b);
		
		// Affichage des résultats
		printf("Coefficients de la régression linéaire :\n");
		printf("a = %.2f\n", a);
		printf("b = %.2f\n\n", b);
		
		// Prédire le prix pour x = 80
		int x_predict = 80;
		float predicted_y = predictY(a, b, x_predict);
		
		printf("Le prix prédit pour une surface de %d m² est : %.2f MGA\n", x_predict, predicted_y);
		
		return 0;
	}
\end{lstlisting}

\subsection*{Explication du Code}

Le programme calcule les coefficients \(a\) et \(b\) de la régression linéaire en utilisant les formules suivantes :
\[
a = \frac{n \sum x_i y_i - \sum x_i \sum y_i}{n \sum x_i^2 - (\sum x_i)^2}
\]
\[
b = \frac{\sum y_i - a \sum x_i}{n}
\]
Ensuite, il utilise ces coefficients pour prédire \(y\) pour \(x = 80\).

\subsection*{Résultats}

En exécutant ce programme avec les données fournies, le programme affiche les coefficients de régression et la prédiction pour \(x = 80\).



\section*{Régression Logistique}


\subsection*{Implémentation en C}

Voici l'implémentation en C pour résoudre l'exercice :

\begin{lstlisting}[caption=Implémentation de la Régression Logistique en C]
	#include <stdio.h>
	#include <math.h>
	
	// Fonction pour calculer z
	float computeZ(float beta0, float beta1, float x1) {
		return beta0 + beta1 * x1;
	}
	
	// Fonction sigmoïde pour transformer z en probabilité P
	float sigmoid(float z) {
		return 1 / (1 + exp(-z));
	}
	
	int main() {
		// Données d'exemple
		float beta0 = 0.5;  // Coefficient intercept
		float beta1 = 0.1;  // Coefficient pour x1 (temps)
		float x1 = 12;      // Temps en minutes
		
		// Calcul de z
		float z = computeZ(beta0, beta1, x1);
		
		// Calcul de la probabilité P
		float P = sigmoid(z);
		
		// Affichage des résultats
		printf("Valeur de z : %.2f\n", z);
		printf("Probabilité P : %.2f\n", P);
		
		return 0;
	}
\end{lstlisting}

Le programme utilise la fonction sigmoïde pour transformer la valeur \( z \) calculée en probabilité \( P \), qui peut ensuite être utilisée pour prendre des décisions dans un modèle de régression logistique.

\subsection*{Explication}

Le programme calcule d'abord \( z \) en utilisant la formule de la régression linéaire :

\[
z = \beta_0 + \beta_1 \cdot x_1
\]

Ensuite, il applique la fonction sigmoïde :

\[
P = \frac{1}{1 + e^{-z}}
\]

Cela donne la probabilité \( P \) associée à une certaine valeur \( x_1 \) (dans ce cas, 12 minutes).


\section*{ K-means}

Voici un programme en C qui résout l'exercice demandé. Le programme suit les étapes suivantes :
\begin{itemize}
	\item Accepter une liste de points \( (x, y) \).
	\item Initialiser 2 centroïdes aléatoires.
	\item Assigner chaque point au centroïde le plus proche.
	\item Recalculer les centroïdes et afficher les groupes finaux.
\end{itemize}

\subsection*{Données}

Les données d'entrée sont les suivantes :

\begin{table}[h]
	\centering
	\caption{Données pour K-means}
	\begin{tabular}{|c|c|}
		\hline
		Point & Coordonnées $(x, y)$ \\ \hline
		A     & (1, 2)               \\ \hline
		B     & (3, 4)               \\ \hline
		C     & (5, 6)               \\ \hline
		D     & (8, 9)               \\ \hline
	\end{tabular}
\end{table}

\subsection*{Implémentation en C}

Voici l'implémentation en C pour résoudre l'exercice :

\begin{lstlisting}[caption=Implémentation de l'algorithme K-means en C]
	#include <stdio.h>
	#include <math.h>
	#include <stdlib.h>
	
	// Structure pour un point (x, y)
	typedef struct {
		float x;
		float y;
	} Point;
	
	// Fonction pour calculer la distance entre deux points
	float distance(Point a, Point b) {
		return sqrt(pow(a.x - b.x, 2) + pow(a.y - b.y, 2));
	}
	
	// Fonction pour initialiser les centroïdes aléatoires
	void initializeCentroids(Point points[], int k, Point centroids[]) {
		centroids[0] = points[0];  // Initialisation manuelle pour le premier centroïde
		centroids[1] = points[3];  // Initialisation manuelle pour le deuxième centroïde
	}
	
	// Fonction pour assigner chaque point au centroïde le plus proche
	void assignPointsToCentroids(Point points[], int n, Point centroids[], int clusters[]) {
		for (int i = 0; i < n; i++) {
			float minDistance = distance(points[i], centroids[0]);
			clusters[i] = 0;
			
			for (int j = 1; j < 2; j++) {
				float dist = distance(points[i], centroids[j]);
				if (dist < minDistance) {
					minDistance = dist;
					clusters[i] = j;
				}
			}
		}
	}
	
	// Fonction pour recalculer les centroïdes
	void recalculateCentroids(Point points[], int n, Point centroids[], int clusters[]) {
		float sumX[2] = {0, 0}, sumY[2] = {0, 0};
		int count[2] = {0, 0};
		
		for (int i = 0; i < n; i++) {
			sumX[clusters[i]] += points[i].x;
			sumY[clusters[i]] += points[i].y;
			count[clusters[i]]++;
		}
		
		for (int i = 0; i < 2; i++) {
			centroids[i].x = sumX[i] / count[i];
			centroids[i].y = sumY[i] / count[i];
		}
	}
	
	int main() {
		// Données des points
		Point points[] = {{1, 2}, {3, 4}, {5, 6}, {8, 9}};
		int n = 4;  // Nombre de points
		
		// Initialisation des centroïdes
		Point centroids[2];
		initializeCentroids(points, 2, centroids);
		
		// Assignation des points aux centroïdes
		int clusters[4];
		assignPointsToCentroids(points, n, centroids, clusters);
		
		// Affichage des groupes après la première assignation
		printf("Assignation des points aux clusters :\n");
		for (int i = 0; i < n; i++) {
			printf("Point (%f, %f) -> Cluster %d\n", points[i].x, points[i].y, clusters[i]);
		}
		
		// Recalcul des centroïdes
		recalculateCentroids(points, n, centroids, clusters);
		
		// Affichage des nouveaux centroïdes
		printf("\nNouveaux centroïdes :\n");
		for (int i = 0; i < 2; i++) {
			printf("Centroïde %d : (%f, %f)\n", i + 1, centroids[i].x, centroids[i].y);
		}
		
		return 0;
	}
\end{lstlisting}

Le programme fonctionne comme suit :
1. Il initialise deux centroïdes à partir des points donnés (ici, manuellement sélectionnés pour simplifier).
2. Chaque point est assigné au centroïde le plus proche.
3. Les centroïdes sont recalculés en prenant la moyenne des points de chaque groupe.
4. Enfin, les résultats sont affichés, montrant les groupes de points ainsi que les nouveaux centroïdes.

\subsection*{Explication des résultats}

Après avoir exécuté le programme, vous obtiendrez l'assignation des points à leurs clusters respectifs et les nouveaux centroïdes recalculés. L'algorithme peut être itéré plusieurs fois pour améliorer les résultats.



 % Solutions for exercises
	
	%\nocite{*}
	%\bibliographystyle{plain}
	%\bibliography{reference}	
\end{document}
